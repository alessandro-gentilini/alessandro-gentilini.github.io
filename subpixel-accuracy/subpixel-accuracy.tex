\documentclass[a4paper,12pt]{article}
\usepackage[utf8]{inputenc}
\usepackage[english]{babel}

\usepackage{siunitx}
\usepackage{graphicx}
\usepackage{amsmath}
\usepackage{hyperref}
\hypersetup{
    colorlinks=false,
    pdfborder={0 0 0},
}

\usepackage[backend=bibtex]{biblatex}
\bibliography{bibliography.bib}



\title{Some notes about subpixel accuracy}
\author{Alessandro Gentilini\thanks{alessandro.gentilini@gmail.com}}


\begin{document}
\maketitle

\begin{abstract}
A bibliography for the problem of non-terminating game of Beggar-My-Neighbour.
\end{abstract} 

\section{A bibliography for the problem of non-terminating game of 
Beggar-My-Neighbour}

In \cite[section 8.2, p.270]{Jahne:2004:PHI:983100}:
\begin{quotation}
In order to transform images back from pixel coordinates to world coordinates,
various transformations are required. In the simplest case, this
includes only scaling,
translation (shifting), and rotation of images. More general are
affine (Section 8.3.1b)
and perspective (Section 8.3.1c) transformations. For precise
geometric measurements,
it is also required to correct for the residual geometric distortion
introduced by even
well-corrected optical systems. Modern imaging solid-state sensors are
geometrically
very precise and stable. Therefore, the potential of a position
accuracy of better than
1/100 pixel distance is there. To maintain this accuracy, all
geometric transformations
applied to digital images must preserve this high position accuracy.
This demand goes
far beyond the fact that no distortions are visible in the transformed images.
\end{quotation}

In \cite{halcon}
\begin{quotation}
HALCON’s 3D calibration supports both area and line scan cameras and
permits, for example, subpixel-accurate measurements up to \SI{1}{\micro\metre} in a field of
view of 10 mm.
\end{quotation}

In \cite{Heikkila:2000:GCC:354167.354171}
\begin{quotation}
Modern CCD cameras are usually capable of a spatial accuracy greater than 1/50 of the pixel
size. However, such accuracy is not easily attained due to various error sources that can affect
the image formation process.
\end{quotation}

In \cite{Fisher}
\begin{quotation}
A commonly quoted rule of thumb is 0.1 pixel, but lower is achievable, e.g. about 0.02 pixel is shown for stripe position detection in [1]\footnote{As \cite{doi:10.1117/12.55947}.}.
\end{quotation}

\printbibliography
\end{document}
