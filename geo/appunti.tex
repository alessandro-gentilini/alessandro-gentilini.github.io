\documentclass[a4paper]{article}
\usepackage[T1]{fontenc}
\usepackage[utf8]{inputenc}
\usepackage[italian]{babel}

\usepackage[hidelinks]{hyperref}
\usepackage[italian]{isodate}
\usepackage{natbib}

\usepackage{newpxtext}
\usepackage{euler}

\usepackage{amsmath}

\begin{document}
\title{Appunti}
\author{Alessandro Gentilini}
\maketitle

Tratto dalla \textit{Piccola Enciclopedia Treccani}, 1995.


\textbf{onda} Con riferimento a un dato mezzo fisico (acqua, aria ecc.), perturbazione determinatasi in un punto del mezzo che si propaga nello spazio trasportando energia ma non materia. 

\section{Propagazione per onde}

Si parla di propagazione per o. di una perturbazione tutte le volte che in uno o più punti di un corpo, o più genericamente di un mezzo o di un ente fisico, si determina una perturbazione di qualche caratteristica (per es., in un corpo elastico uno spostamento di suoi punti dalle loro posizioni di riposo) e questa perturbazione si trasmette dai punti dove si è generata a zone a essi circostanti, trasportando energia ma non materia. 

La sorgente da cui nasce la perturbazione può essere estesa o (approssimativamente) puntiforme. 

I punti raggiunti dalla perturbazione a un dato istante costituiscono generalmente una superficie che prende il nome di superficie d'o., o \textit{fronte d'o.}, all'istante considerato. La direzione di propagazione in un dato punto $P$ di un fronte d'o. è la direzione in cui si propaga una porzione infinitesima intorno a $P$ del fronte stesso. 

Il raggio di propagazione dell'o. (o più semplicemente il \textit{raggio}) è la linea che in ogni punto è tangente alla direzione di propagazione di una data porzione infinitesima del fronte d'onda. Non è detto che i raggi siano rettilinei: lo sono nel caso in cui le superfici d'o. siano costituite da sfere concentriche (o. sferiche) e quindi i raggi di propagazione siano i raggi stessi di queste sfere; come anche in quello in cui le superfici siano piani paralleli (o. piane) e i raggi quindi siano le rette, parallele, normali a tali piani. 

Se la perturbazione associata all'o. può essere individuata da un vettore, l'orientamento di questo rispetto al raggio può essere a priori qualsiasi: particolare importanza ha il caso che il vettore della perturbazione abbia direzione coincidente con la direzione di propagazione (o. longitudinali) e quello che esso risulti normale alla direzione di propagazione (o. trasversali): per es., il suono si propaga negli aeriformi per o. longitudinali, la luce per o. trasversali. 

Lungo ogni direzione la propagazione può, a priori, avvenire in un verso o nel verso opposto: così l'o. si dice progressiva o retrograda a seconda che la propagazione avvenga nel verso fissato come positivo sulla direzione considerata o nel verso opposto, come verso positivo assumendosi di solito quello di allontanamento dalla sorgente. Un'o. si dice poi permanente se le sue caratteristiche rimangono inalterate, indefinitamente, lungo ogni raggio. Un'o. permanente è peraltro una pura astrazione poiché in pratica l'o. è sempre smorzata in conseguenza della distribuzione su fronti d'o. sempre più ampi e dell'eventuale dissipazione di una parte dell'energia che essa convoglia. Dalla sovrapposizione di o. che si propaghino lungo una stessa direzione può non risultare una propagazione ondosa: è il caso, per es., delle cosiddette o. stazionarie. 

Numerosi sono i fenomeni fisici con carattere ondoso, e dalla natura di ciascuno di essi si denominano le onde. Una prima grande ripartizione può farsi tra o. elastiche e o. elettromagnetiche: costituite, le prime, da una perturbazione oscillatoria delle particelle materiali di un mezzo elastico, e le seconde dalla propagazione, in un mezzo qualunque o nel vuoto, di un campo elettrico e di un campo magnetico, entrambi variabili. In seno poi a queste due principali categorie ulteriori distinzioni possono farsi in relazione a caratteristiche intrinseche dell'o., per es., in base al carattere longitudinale o trasversale, progressivo o stazionario, in base alla forma dei fronti d'o., alla frequenza ecc., ovvero in base alle caratteristiche del mezzo. 

\section{Onde nelle corde elastiche vibranti}
Si consideri un filo elastico (corda elastica) omogeneo, preliminarmente disposto secondo la retta $x$, soggetto a una tensione costante, che a un certo istante prenda a vibrare, per l'intervento di opportune sollecitazioni, in modo che i suoi elementi compiano piccole oscillazioni in un determinato piano $\alpha$ (fig. 1) individuato dalla retta $x$ (che si assume come asse delle ascisse, fissando su essa un'origine e un verso positivo di percorrenza) e da una direzione $\mathbf{n}$ ortogonale a essa. Se $x$ è l'ascissa del generico elemento della corda nella sua originaria configurazione rettilinea, e $t$ è il tempo, lo spostamento trasversale $s$ che istante per istante subisce l'elemento suddetto è una funzione di $x$ e $t$, che, come si dimostra a partire dalla seconda legge della dinamica, deve soddisfare l'equazione alle derivate parziali del secondo ordine
\begin{equation}
\label{eqn:1}
equazione delle onde
\end{equation}

(dove le costanti $\rho$ e $\tau$ rappresentano rispettivamente la massa per unità di lunghezza, o densità lineica, della corda e la tensione), nonché a determinare condizioni al contorno, variabili da caso a caso (corda fissata rigidamente a un solo estremo, fissata elasticamente ai due estremi ecc.), e a determinate condizioni iniziali (posizione e velocità iniziale dei singoli punti della corda). 
Per soddisfare la \eqref{eqn:1}, 
$s$ dovrà risultare, per ogni elementino (o particella), piccolo di fronte alla lunghezza $l$ della corda, ciò che si può tradurre nella condizione che il rapporto $s/l$ sia abbastanza piccolo perché se ne possano trascurare le potenze superiori alla prima. La \eqref{eqn:1} è nota come equazione delle corde vibranti o equazione di d'Alembert; la più generale soluzione di essa è della forma (integrale di d'Alembert) 
\begin{equation}
\label{eqn:2}
s(x,t)=s_1(x-vt)+s_2(x+vt)
\end{equation}
dove $s_1$ e $s_2$ sono funzioni, a priori arbitrarie, degli argomenti $x-vt$ e $x+vt$, e $v$ è una costante, avente le dimensioni di una velocità, legata a $\tau$ e $\rho$ dalla relazione 
\begin{equation*}
v=\sqrt{\frac{\tau}{\rho}}
\end{equation*}

Da quest'ultima relazione segue che la \eqref{eqn:1} può anche essere scritta nella forma: 
\begin{equation}
\label{eqn:3}
\frac{\partial^2 s}{\partial x^2}=\frac{1}{v^2}\frac{\partial^2 s}{\partial t^2}
\end{equation}

\subsection{Onde progressive}
Vediamo ora di renderci conto del significato delle due funzioni $s_1$ e $s_2$ che compaiono nella \eqref{eqn:2}. Consideriamo la prima di esse, $s_1$. Si tratta di una funzione il cui argomento è $x-vt$; questo argomento non si altera se si dà a $t$ un qualunque incremento $dt$ e al tempo stesso si incrementa $x$ della quantità $dx=v\,dt$: infatti è $(x + v\,dt) - v(t + dt) = x - vt$.

Se quindi a un dato istante $\bar{t}$ lo spostamento $s_1$ ha un certo valore $\bar{s}$ in un punto $\bar{x}$, all'istante successivo $\bar{t}+dt$ esso avrà lo stesso valore nel punto $\bar{x}+v\,dt$, spostato rispetto a $\bar{t}$ del trattino $v\,dt$ nel verso delle $x$ crescenti. Poiché ciò vale per ogni punto, la configurazione della corda all'istante $\bar{t}+dt$ sarà la stessa che all'istante $\bar{t}$ ma traslata di $v\,dt$ nel verso delle $x$ crescenti. In altri termini, le cose vanno come se la configurazione assunta dalla corda a un dato istante si spostasse, mantenendosi inalterata, con una velocità di propagazione $v$ nel verso positivo dell'asse $x$: $s_1$ rappresenta, con ovvia denominazione, un'o. progressiva permanente (o progressiva non smorzata). Secondo quanto premesso, si tratta di un'o. trasversale, in quanto le vibrazioni degli elementini della corda avvengono tutte secondo la direzione $\mathbf{n}$ ortogonale alla direzione di propagazione $x$: il piano $x\mathbf{n}$, cioè il piano $\alpha$ della fig. 1, si chiama piano di vibrazione, mentre il piano $\beta$ contenente $x$ e normale al piano $\alpha$ si chiama piano di polarizzazione, e l'o. si dice polarizzata. 

\subsection{Onde armoniche}
Particolarmente semplice e importante è il caso delle o. armoniche o \textit{sinusoidali}, caratterizzate dal fatto che la configurazione della corda, o, come si dice, il profilo d'o., è una sinusoide: 
\begin{equation}
\label{eqn:4}
s_1=a\sin [b(x-vt)+\phi]
\end{equation},

dove $a$, $b$, $\phi$ sono opportune costanti, delle quali $a$, che rappresenta lo spostamento massimo del generico elementino della corda dalla sua posizione d'equilibrio, ha il nome di ampiezza dell'onda. La quantità che costituisce l'argomento del seno, cioè $b(x-vt)+\phi$, da misurarsi in radianti, si chiama fase dell'o. nel punto di ascissa $x$ e all'istante $t$; pertanto $\phi$ è la fase all'istante iniziale ($t=0$) nell'origine delle ascisse ($x=0$). Convenendo di indicare con $\lambda$ l'intervallo di ascisse percorrendo il quale la fase dell'o. varia di $2\pi$ rad, si ha $b={2\pi}/\lambda$, di modo che la \eqref{eqn:4} può porsi nella forma: 
\begin{equation}
\label{eqn:5}
s_1=a\sin [\frac{2\pi}{\lambda}(x-vt)+\phi]
\end{equation}

ovvero, ponendo ulteriormente 
\begin{equation}
\label{eqn:6}
T=\frac{\lambda}{v}
\end{equation}

nella forma equivalente 
\begin{equation}
\label{eqn:7}
s_1=a\sin [2\pi(\frac{x}{\lambda}-\frac{t}{T})+\phi]
\end{equation}

Dal fatto che la funzione seno è periodica con periodo $2\pi$ rad, segue che $s_1$ è funzione periodica con periodo 1 della variabile $(x/\lambda-t/T)$, cioè che essa è funzione periodica tanto di $x$ quanto di $t$, rispettivamente con i periodi $\lambda$ e $T$. Ciò significa che, se si fissa l'attenzione su un determinato valore di $x$, per la vibrazione trasversale del corrispondente elementino della corda si ha nel tempo un andamento sinusoidale (fig. 2A), con periodo $T$, che si chiama perciò periodo del moto ondoso in esame: si ha cioè uno stesso valore, $\bar{s_1}$, per lo spostamento relativo a due istanti $\bar{t}$ e $\bar{t} + nT$ (con $n$ intero non nullo) che differiscono tra loro di $T$ o di multipli interi di $T$.
Se invece si fissa l'attenzione su un determinato istante, l'andamento in quell'istante degli spostamenti trasversali delle varie particelle della corda, cioè il profilo dell'o., è anch'esso sinusoidale, con il periodo $\lambda$, che si chiama lunghezza d'o., in quanto se ci si muove lungo $x$ di un tratto pari a $\lambda$ o a un multiplo intero di $\lambda$ (per es., da $\bar{x}$ a $\bar{x}+\lambda$; fig. 2B), gli spostamenti si riproducono identicamente. Com'è chiaro, a norma della \eqref{eqn:6} e della definizione data per $\lambda$, il periodo $T$ rappresenta l'intervallo di tempo in cui l'o. percorre un tratto pari alla sua lunghezza d'o. $\lambda$ e in cui la fase dell'o. varia di $2\pi$ rad; all'inverso del periodo, $\nu$, si dà il nome di frequenza dell'o.: 
\begin{equation}
\label{eqn:8}
\nu=\frac{1}{T}=\frac{v}{\lambda}
\end{equation}

alla quantità
\begin{equation}
\label{eqn:9}
\omega=\frac{2\pi}{T}=2\pi\nu
\end{equation}

che rappresenta, in rad/s, la variazione della fase dell'o. nell'unità di tempo, si dà il nome di pulsazione dell'o.; infine, l'inverso, $k$, della lunghezza d'o., 
\begin{equation}
\label{eqn:10}
k=\frac{1}{\lambda}=\frac{1}{vT}=\frac{\nu}{v}
\end{equation}

che rappresenta il numero (generalmente non intero) di lunghezze d'o. compreso nell'unità di lunghezza, si chiama numero di onde. Grazie alle grandezze ora introdotte, per rappresentare l'o. che si propaga lungo la corda nel verso positivo dell'asse $x$ si possono usare le seguenti espressioni, equivalenti tra loro ed equivalenti alle \eqref{eqn:5}, \eqref{eqn:7}: 
\begin{equation}
s_1=a\sin[2\pi(\frac{x}{\lambda}-\nu t)+\phi]
s_1=a\sin(\frac{2\pi x}{\lambda}-\omega t+\phi)
s_1=a\sin[2\pi(kx-\nu t)+\phi]
\end{equation}

È da notare che con una conveniente scelta dell'istante dal quale si comincia a misurare lo scorrere del tempo, $\phi$ può assumere un qualsivoglia valore: in particolare, si può ottenere che sia $\phi=0$, oppure che la funzione seno venga sostituita dalla funzione coseno o che muti il segno dell'argomento. In certe questioni risulta poi comoda la rappresentazione complessa del moto ondoso, basata sull'osservazione che una qualunque delle espressioni prima date per $s_1$ può essere considerata, in virtù della formula di Eulero, come la parte reale di un'opportuna funzione complessa. Per es., alla \eqref{eqn:10} corrisponde la parte reale della funzione 
\begin{equation}
s_1=a\exp\{i[2\pi(kx-\nu t)+\phi-\frac{pi}{2}]\},
\end{equation},

essendo $i^2=-1$; si può allora scrivere:
\begin{equation}
\{s_1=A\exp[2\pi i(kx-\nu t)]
A=a\exp[i(\phi-\frac{pi}{2})]
\end{equation}.

$A$ è detta ampiezza complessa dell'o.: il suo modulo è l'ampiezza a dell'o., mentre il suo argomento cambiato di segno è la fase iniziale nell'origine. L'importanza delle o. armoniche deriva dal fatto che a esse ci si può in definitiva ricondurre in ogni caso. Per un teorema di Fourier, infatti, la funzione $s$ che compare nella \eqref{eqn:2}, anche se non periodica, può sempre scriversi come una sovrapposizione di funzioni sinusoidali, di modo che un'o. qualunque può sempre pensarsi come risultante dalla sovrapposizione di o. armoniche. 

\subsection{Onde retrograde}
Le stesse considerazioni sin qui svolte a proposito della funzione $s_1(x-vt)$ possono essere ripetute a proposito della funzione $s_2(x+vt)$, salvo che quest'ultima rappresenta un'o. che si propaga nel verso delle $x$ decrescenti, cioè un'o. retrograda; anche per $s_2$ valgono, nel caso di o. armoniche, le espressioni date per $s_1$, salvo il segno contrario per il termine contenente la variabile temporale ($+t/T$ in luogo di $-t/T$; $+\nu t$ in luogo di $-\nu t$; ecc.). La presenza contemporanea nella \eqref{eqn:2} di una funzione rappresentante un'o. progressiva e di una funzione rappresentante un'o. retrograda è cosa che non deve stupire, dipendendo essa dalle particolari condizioni in cui si opera, ovvero, in termini analitici, dalle condizioni iniziali e al contorno da imporre nei vari casi all'equazione \eqref{eqn:1}. 

\subsection{Onde stazionarie}
Un caso notevole di sovrapposizione dell'o. progressiva e dell'o. retrograda è quello che dà luogo alla formazione di o. stazionarie, caratterizzate dalla circostanza che certi punti della corda, detti nodi di vibrazione, in posizione fissa, restano in quiete, mentre in altri punti, detti ventri di vibrazione, anch'essi corrispondenti a particolari valori dell'ascissa $x$, l'ampiezza di vibrazione è massima: il profilo d'o. è quindi immobile. Il fenomeno si produce, per es., in corde di lunghezza finita, fissate agli estremi. Il corrispondente problema analitico consiste nel risolvere la \eqref{eqn:3} sotto la duplice condizione che sia, per qualunque valore di $t$, $s=0$ per $x=0$ e per $x=l$, convenendo di assumere l'origine delle $x$ in un estremo e di indicare con $l$ la lunghezza della corda nella sua configurazione d'equilibrio, rappresentata dall'asse $x$; una soluzione particolare è: $s=a \sin px \sin{pvt+q}$, in cui $a$, $p$, $q$ sono costanti da determinare in base alle condizioni al contorno (per $x$) e iniziali (per $t$). Imponendo le condizioni prima precisate per $x$ si ha per la soluzione generale: 
\begin{equation}
\label{eqn:11}
s=\sum_{r=1}^{+\infty}a_r\sin\frac{r\pi x}{l}\cos(\frac{r\pi vt}{l}+\phi_r)
\end{equation},

in cui $r$ è un intero positivo, non nullo. Tale relazione mostra come nella corda si possano destare infinite o. stazionarie, ciascuna corrispondente a un valore di $r$, che hanno il nome di modi di vibrazione; $a_r$ e $\phi_r$, ampiezza e fase iniziale del modo $r$-esimo, risultano determinate in base alle condizioni iniziali, le quali per di più determinano anche quali sono i modi di vibrazione che effettivamente si destano nella corda. Che ciascun termine della \eqref{eqn:11} sia un'o. stazionaria risulta chiaramente dal fatto che nei punti la cui ascissa $x$ vale ${nl}/r$, con $n=0, 1,\ldots, r$, l'ampiezza è nulla, cioè si hanno nodi, mentre nei punti di ascissa $x=(2n+1)l/{2r}$, con $n=0, 1,\ldots, r-1$, l'ampiezza è massima e pari ad $a_r$, cioè si hanno ventri. Si ricava per la pulsazione del modo $r$-esimo il valore $\omega r=\frac{r\pi v}{l}$, e quindi per la frequenza il valore $\nu r=\frac{rv}{2l}$ e per la lunghezza d'o. il valore $\lambda r=\frac{2l}{r}$. 
Risulta così che per r=1 la frequenza ha il valore più basso possibile e la lunghezza d'o. è la massima possibile: è l'o. fondamentale, o prima armonica, corrispondente al fatto che la corda contiene una semilunghezza d'o., e che il punto di mezzo della corda è l'unico ventre (M in fig. 3A). Per r=2, 3,... si hanno o. di frequenza doppia, tripla, ... rispetto a quella dell'o. fondamentale (seconda armonica, terza armonica, ...), di cui la corda contiene 2, 3,... semilunghezze; per valori pari di r nel punto di mezzo M della corda cade un nodo (fig. 3 B), mentre per r dispari in M cade un ventre (fig. 3 C). La soluzione adatta al caso che si sta studiando si ottiene imponendo alla \eqref{eqn:11} le condizioni iniziali, cioè le condizioni che danno conto della posizione iniziale e della velocità iniziale di ogni punto della corda. Queste grandezze dipendono essenzialmente dal modo con cui la corda viene eccitata a vibrare. Per es., se la corda viene eccitata nel suo punto di mezzo, ar risulta nulla per r pari, e inversamente proporzionale a r2 per r dispari: la corda vibra essenzialmente sulla frequenza fondamentale. Il fatto che si possano determinare vibrazioni più o meno ricche di armoniche viene utilizzato in musica per gli strumenti a corda. 

\subsection{Onde smorzate} 
Abbiamo sin qui considerato o. di ampiezza costante, ma in realtà la presenza di inevitabili resistenze passive (se non altro, la resistenza offerta dall'aria in cui la corda vibra) fa sì che le vibrazioni siano in genere smorzate. Come si sa dalla meccanica, ciò comporta la presenza di un termine contenente la derivata prima $\frac{\partial s}{\partial t}$ nella \eqref{eqn:3}, la quale assume la forma: 
\begin{equation}
\frac{\partial^2 s}{\partial x^2}=\frac{1}{v^2}(\frac{\partial^2 s}{\partial t^2}+\alpha\frac{\partial s}{\partial t})
\end{equation}

Come si dimostra, per un'o. armonica progressiva la soluzione è del tipo:
\begin{equation}
s = a \exp(- \alpha \frac{t}{2})\sin[b (x - vt) +\phi]
\end{equation},

dove $a$, $b$, $\phi$ sono costanti da determinare e $\frac{2}{\alpha}$ è la costante di tempo dello smorzamento della vibrazione. 

\section{Onde in un generico mezzo elastico}
Come in una corda elastica, così in un generico mezzo elastico si possono determinare fenomeni ondosi. Precisamente, se in un punto di un sistema elastico (solido, liquido o aeriforme) si determina, per una causa qualsiasi, una perturbazione di preesistenti condizioni di equilibrio, generalmente tale perturbazione non si limita all'elemento in cui ha avuto origine, ma si trasmette agli elementi immediatamente vicini. Ciascuno di questi a sua volta lo trasmette agli elementi circostanti: la perturbazione si propaga così da un punto all'altro del sistema, in tutte le direzioni accessibili, con più o meno grande velocità. Il problema dinamico che si pone, in relazione alla propagazione di tale perturbazione, è ancora quello di determinare in funzione del tempo $t$ lo spostamento (vettoriale) $\mathbf{s}$ del generico elementino del sistema, quando siano note le forze e assegnate le condizioni al contorno e iniziali (posizione e velocità) di ciascun elementino. Il problema si traduce in un sistema di tre equazioni differenziali alle derivate parziali del second'ordine nelle componenti cartesiane $s_x$, $s_y$, $s_z$ di $\mathbf{s}$, equazioni che subito si ottengono da quelle della statica elastica sostituendo, a norma del principio di d'Alembert, alle forze attive le forze perdute.

\subsection{Onde elastiche irrotazionali}
Nel caso delle o. elastiche irrotazionali, che si ha quando lo spostamento s è un vettore irrotazionale, ossia è rots=0, oppure, in termini equivalenti, s deriva da un potenziale $\phi$, se si trascurano le resistenze passive, il moto è retto da un'equazione scalare che si può porre nella forma 
\begin{equation}
\nabla^2\phi=\frac{1}{v^2}\frac{\partial^2 \phi}{\partial t^2}
\end{equation}

dove $\nabla^2$ è il laplaciano e v è la velocità di propagazione, legata alle costanti di Lamé, $\lambda$ e $\mu$, e alla densità $\rho$ del mezzo (supposto omogeneo e isotropo) dalla relazione 
\begin{equation}
v=\sqrt{\frac{\lambda+2\mu}{\rho}}
\end{equation}

La [13] è dello stesso tipo della \eqref{eqn:1}, e ad analoghe equazioni soddisfano le componenti dello spostamento s; quest'ultimo si ottiene, ovviamente, calcolando il gradiente di $\phi$. Se $\phi$ dipende, oltre che dal tempo, da una soltanto delle tre coordinate spaziali, per es. dalla x, si ha:
\begin{equation}
\frac{\partial^2 s_x}{\partial x^2}=\frac{1}{v^2}\frac{\partial^2 s_x}{\partial t^2}
\end{equation},

mentre le altre due componenti sy e sz di s risultano identicamente nulle. La soluzione generale della [15] è, naturalmente, dello stesso tipo della \eqref{eqn:2}
\begin{equation}
sx = sx1(x - vt) + sx2(x + vt)
\end{equation},

dove sx1, sx2 sono funzioni determinabili in base alle condizioni al contorno e iniziali. Le superfici (o fronti) d'o., ognuna delle quali è il luogo dei punti in cui l'argomento della funzione rappresentante l'o. progressiva, x-vt, ha uno stesso valore (e analogamente per la funzione rappresentante l'o. retrograda), sono piani normali a x, in ciascuno dei quali lo spostamento, parallelo a x, ha ovviamente lo stesso valore: la [16] rappresenta o. irrotazionali piane longitudinali, la qualifica di piane riferendosi alla forma della superficie d'o. e quella di longitudinali al fatto che la vibrazione del generico elementino del mezzo avviene lungo la direzione di propagazione, x. Può anche darsi il caso che le superfici d'o. non siano piane.

Tra le o. non piane, particolare importanza hanno le o. sferiche, che si considerano quando la propagazione ondosa abbia simmetria sferica rispetto a un punto O: caso che effettivamente si ha quando la sorgente della perturbazione è puntiforme e il mezzo è omogeneo e isotropo. Il gradiente di $\phi$ dà lo spostamento s, e la divergenza di quest'ultimo dà la dilatazione (o compressione) volumica $\gamma$ del mezzo; si ha dunque $\gamma=div s=div grad \phi=\nabla^2 \phi$, e ciò significa che $\gamma$ varia nello stesso modo con cui varia lo spostamento: corrispondentemente a spostamenti del generico elementino del mezzo da una parte e dall'altra rispetto alla posizione d'equilibrio, il mezzo si addensa e si rarefà, cioè nel mezzo medesimo si propagano o. di compressione e di dilatazione. 

\subsection{Onde elastiche rotazionali}
Sono o. che si producono in un mezzo elastico incomprimibile, in cui cioè si abbia $\gamma=div s=0$, essendo $\gamma$ la dilatazione cubica e s lo spostamento del generico elementino del mezzo. Ponendo 
\begin{equation}
v=\sqrt{\frac{\mu}{\rho}}
\end{equation}

ove i simboli hanno il significato precedentemente indicato, si riconosce che lo spostamento deve soddisfare all'equazione vettoriale 
\begin{equation}
\nabla^2 s =\frac{1}{v^2}\frac{\partial^2 s}{\partial t^2}
\end{equation}

che, proiettata sugli assi, dà luogo a tre equazioni formalmente analoghe a quelle che si hanno nel caso delle o. irrotazionali, ma rappresentanti un fenomeno di propagazione ondosa del tutto diverso. Lo spostamento, infatti, che nelle o. irrotazionali avviene lungo la direzione di propagazione, avviene ora perpendicolarmente a tale direzione: di qui la qualifica di o. trasversali. Inoltre, la velocità di propagazione, data dalla [17], è, a parità di ogni altra condizione, minore di quella delle o. irrotazionali, data dalla [14]. Analogamente al caso delle o. irrotazionali, può accadere che il fronte d'o. sia piano (o. piane trasversali) oppure sferico (o. sferiche trasversali); se in particolare la direzione di propagazione coincide con l'asse x del sistema cartesiano di riferimento, il caso delle o. piane corrisponde alla circostanza che s ha identicamente nulla la componente sx (s è normale a x), mentre le componenti sy e sz dipendono unicamente da x e dal tempo t. In particolari condizioni può accadere che il rapporto $sx/sy=tg \theta$ sia costante; ciò significa che s giace sempre nel piano (detto piano di vibrazione) contenente l'asse x e inclinato dell'angolo $\theta$ rispetto al piano xy: l'o. si dice polarizzata rettilineamente, e si chiama piano di polarizzazione di essa il piano contenente l'asse x e perpendicolare al piano di vibrazione. Se $\theta=0$, sz è identicamente nulla (il piano di vibrazione è il piano xy), mentre se $\theta=\frac{\pi}{2}$, è identicamente nulla sy (il piano di vibrazione è il piano xz). Il fenomeno della polarizzazione è esclusivo delle o. trasversali: esso cioè non può verificarsi per le o. longitudinali. 

\subsection{Onde elastiche di tipo generale}
Una generica o. elastica si può sempre considerare come dovuta alla combinazione di un'o. irrotazionale e di un'o. rotazionale; quel che v'è di importante è che essa, se permanente, è sempre rappresentata dall'equazione 
\begin{equation}
\nabla^2 K = \frac{1}{v^2}\frac{\partial^2 K}{\partial t^2}
\end{equation}

di cui in precedenza sono state esaminate varie forme particolari; a seconda delle circostanze, K può essere una grandezza scalare oppure una grandezza vettoriale, come pure, a seconda delle circostanze, variano le condizioni al contorno e iniziali, le quali in particolare determinano il ‘carattere' dell'o., che può risultare progressiva o retrograda, oppure una combinazione di questi due tipi e in particolare stazionaria, e inoltre può risultare un'o. a una dimensione (com'è, per es., nel caso delle corde vibranti), a due dimensioni (per es., nel caso di membrane vibranti), a tre dimensioni (per es., o. longitudinali nell'aria); in quest'ultimo caso poi, in relazione alla forma del fronte d'o. si possono avere vari tipi di onde. La [18] si presta a descrivere onde di qualunque natura, anche non elastiche; per tale sua proprietà è denominata equazione delle o., o equazione del moto ondoso o equazione della propagazione per onde. In realtà, la [18] non rappresenta con piena generalità la propagazione per o.: essa si riferisce, come abbiamo a suo tempo precisato, a moti ondosi permanenti, non smorzati. A rigore, come equazione generale del moto ondoso va assunta la seguente:
\begin{equation}
\nabla^2 K=\frac{1}{v^2}(\frac{\partial^2 K}{\partial t}+\alpha\frac{\partial K}{\partial t})
\end{equation}

che, mediante il termine $\alpha\frac{\partial K}{\partial t})$, dà conto di un eventuale smorzamento del moto e di cui la [12] costituisce la particolarizzazione nel caso delle corde elastiche vibranti.

È stato ricordato come un caso notevole di propagazione per o. sia costituito dalle o. armoniche, rappresentabili mediante funzioni sinusoidali. In effetti si tratta però di un caso piuttosto particolare, dato che in genere il profilo di un'o. non è sinusoidale, e l'o. medesima è rappresentabile come combinazione lineare di più o. armoniche, con ampiezza, frequenza e fase iniziale opportune. Poiché le grandezze da cui dipende la velocità di propagazione v non sono in genere indipendenti dalla frequenza, le varie componenti armoniche di un'o. non armonica si propagano con velocità diversa l'una dall'altra (fenomeno della dispersione) e il profilo dell'o. cambia di forma via via che l'o. avanza. Si deve allora distinguere tra la velocità v di propagazione di ogni componente dell'o., che chiameremo velocità di fase di quella componente, e la velocità u con cui procede l'o. nel suo insieme, che chiameremo velocità di gruppo; la velocità di fase di una componente è la velocità con cui avanza ogni fronte d'o. di quella componente, mentre u è la velocità di propagazione dell'energia associata all'o. (si tratta di un'energia ripartita tra le varie componenti). La relazione generale tra le due velocità è piuttosto complessa. Nel caso che le componenti siano di frequenza non molto diversa, vale la relazione (formula di Rayleigh): 
\begin{equation}
u=v-\lambda \frac{dv}{d\lambda}
\end{equation}

in cui v è la velocità di fase dell'o. la lunghezza della quale è $\lambda$. 

\subsection{Onde elastiche nei gas}
In seno a un gas possono generarsi e propagarsi o. elastiche longitudinali. La trattazione di tali o. può farsi con riferimento allo spostamento s del generico elementino del mezzo oppure con riferimento al potenziale dello spostamento o della velocità di spostamento. Se ci si riferisce allo spostamento e si assume l'asse x coincidente con la direzione di propagazione, potremo considerare lo spostamento come una funzione scalare, s, di x e del tempo t, che deve soddisfare all'equazione 
\begin{equation}
v^2=\frac{dp}{d\rho}
\end{equation}

essendo p la pressione e $\rho$ la densità. Queste due grandezze non sono indipendenti tra loro; precisamente, se il gas può essere assimilato a un gas perfetto e la frequenza delle o. è relativamente piccola, di modo che le compressioni e le dilatazioni del gas avvengano lentamente e siano assimilabili a trasformazioni isoterme, si ha 
\begin{equation}
p=k\rho
\end{equation}
\begin{equation}
v=sqrt{\frac{p}{\rho}}
\end{equation}

essendo k una costante caratteristica del gas in questione. Se invece la frequenza, come accade nella maggior parte dei casi, è relativamente grande, si può ritenere di essere in presenza di trasformazioni adiabatiche e si può scrivere: 
\begin{equation}
p=k'\rho \gamma
v=\sqrt{\frac{p\gamma}{\rho}}
\end{equation}

essendo $\gamma$ il rapporto tra il calore specifico a pressione costante e il calore specifico a volume costante.

\section{Onde elastiche nei liquidi}
Anche nei liquidi, come è ben noto, si possono generare delle o., per azione del vento, di correnti, di corpi immersi ecc. Si può aggiungere anzi che, storicamente, sono state proprio queste o. (e in special modo quelle marine) a fornire modelli, immagini e nomi per la descrizione e lo studio degli altri fenomeni ondulatori. Si tratta generalmente di o. delle quali non è facile dare soddisfacenti ed esaurienti schematizzazioni matematiche, poiché la semplicità di queste mal si presta a riprodurre la complessità del fenomeno fisico: possono infatti aversi sia o. longitudinali e trasversali, interessanti l'intero volume liquido, sia o. quasi superficiali. Tra le o. suscettibili d'una semplice schematizzazione sono le cosiddette o. progressive permanenti: si tratta in sostanza di quei moti dei liquidi nei quali la forma della superficie libera, o pelo libero, si sposta senza subire modificazioni con una certa velocità di propagazione v costante. È, per es., il caso che si verifica in un canale, a fondo orizzontale e a pareti verticali, di lunghezza molto grande (teoricamente infinita), quando, essendo trascurabile rispetto all'accelerazione di gravità l'accelerazione verticale dei singoli elementini (o particelle) del liquido, ciascuno di essi compie, al progredire dell'o., piccole oscillazioni. Le o. progressive permanenti, se si esclude il caso dell'o. solitaria, sono periodiche, cioè il pelo libero è costituito da archi di curva uguali riproducentisi periodicamente a distanza costante $\lambda$ (lunghezza d'o.); e fra esse vanno particolarmente ricordate le o. trocoidali di F.J. von Gerstner e le o. semplici o sinusoidali di G.B. Airy. Tali denominazioni derivano dalla forma del profilo del pelo libero, che è appunto una trocoide (o, più esattamente, una cicloide accorciata), in un caso, una sinusoide nell'altro. Altre sostanziali differenze tra i due tipi di o. sono: a) si hanno o. di Airy se la profondità h del liquido è molto minore di $\lambda$ (di qui l'altra denominazione di o. di acqua bassa) e o. di Gerstner invece, se $h>>\lambda$ (o. di acqua alta o profonda); b) nelle o. di Airy la perturbazione ondosa interessa tutta la sezione del liquido, mentre nelle o. di Gerstner la perturbazione non si estende molto al di sotto del pelo libero (o. superficiali). Per quanto riguarda le o. di Airy, se si indica con s (fig. 4), lo spostamento del generico elementino (o particella) del pelo libero dalla sua posizione d'equilibrio, assumendosi quest'ultima giacente nel piano xy e l'asse x normale alla sezione del canale, si trova che s deve soddisfare all'equazione
\begin{equation}
\frac{\partial^2 s}{\partial x^2}=\frac{1}{v^2}\frac{\partial^2 s}{\partial t^2}
\end{equation}

con
\begin{equation}
v^2=\frac{Ag}{b}
\end{equation}

g è l'accelerazione di gravità, mentre b e A, generalmente dipendenti da x, sono la larghezza al pelo libero e l'area della sezione verticale ‘liquida' del canale. Se quest'ultima è rettangolare costante, si ha A=bh, essendo h la profondità del liquido, e per la velocità di propagazione si ha: 
\begin{equation}
v=sqrt{gh}
\end{equation}

In generale, per la velocità v vale la relazione: 
\begin{equation}
v=\sqrt{\frac{g\lambda}{2\pi} tgh \frac{2\pi h}{\lambda}}
\end{equation}

Se $h>>\lambda$, è $tgh(2\pi h/\lambda)\cong 1$ e si ha $v=(g\lambda/2\pi )1/2$ (formula di Gerstner); se invece è $h<<\lambda$, si ha $tgh(\frac{2\pi h}{\lambda})\cong \frac{2\pi h}{\lambda}$ e per v si ottiene la [21]. La [22], che costituisce dunque una relazione abbastanza generale, da cui si possono derivare espressioni valide per o. sia di acqua alta sia di acqua bassa, è nota come formula di Airy. Sia le o. di Gerstner sia quelle di Airy sono modelli analitici, che trovano tuttavia largo impiego nelle applicazioni idrauliche e nautiche come base teorica per affrontare complessi problemi pratici. 

\section{Onde elastiche nei solidi}
Ancora più complesso rispetto al caso delle o. nei liquidi è quello delle o. che possono generarsi e propagarsi in un solido; si tratta di o. elastiche di tipo generale, cioè rotazionali o irrotazionali, che spesso si combinano in modi complicati. Le trattazioni analitiche di cui si dispone sono relative a casi particolari; tra questi vanno ricordati: a) il caso delle corde vibranti; b) il caso di membrane elastiche vibranti, la cui trattazione è per molti versi analoga a quella delle o. superficiali nei liquidi; c) il caso di sbarre elastiche, in cui, a differenza dei casi precedenti, si possono destare anche o. longitudinali, la cui trattazione è analoga, almeno formalmente, a quella delle o. in un gas racchiuso in un condotto; d) il caso di un mezzo semiinfinito, studiato da lord Rayleigh, sulla superficie del quale si possono destare o., dette o. di Rayleigh, consistenti in contemporanee vibrazioni trasversali e longitudinali, la cui energia è confinata in uno spessore dell'ordine di una lunghezza d'o. al di sotto della superficie limite del mezzo materiale (l'alta densità di energia elastica collegata con questo tipo di o. dà luogo a vistosi fenomeni di nonlinearità elastica, quali la generazione di armoniche e il mescolamento di frequenze); e) il caso di un mezzo stratiforme, compreso tra altri mezzi di diverse caratteristiche elastiche, nel quale possono insorgere o. di tipo speciale (o. di Love), trasversali, polarizzate, vibranti in un piano orizzontale, la cui propagazione, in particolare, può essere ristretta, in virtù di fenomeni di riflessione totale, allo strato in questione, che agisce come una sorta di guida per le o. medesime (o. canalizzate o guidate). 

\section{Tipi particolari di o. elastiche}

\subsection{O. acustiche}
O. elastiche la cui frequenza è tale da eccitare sensazioni acustiche: precisamente, si tratta di o. la cui frequenza va da circa 16 a circa 20.000 Hz. 

\subsection{O. di Alfvén} 
Tipo di o. magnetoidrodinamiche. 

\subsection{O. di discontinuità}
Con riferimento a un determinato fenomeno (precisamente a fenomeni che si traducano matematicamente in sistemi normali di equazioni alle derivate parziali), o. mediante le quali si propagano nello spazio certe discontinuità di qualche grandezza legata al fenomeno in esame. Si tratta di quei casi in cui esiste una superficie $\sigma$, mobile e generalmente deformabile nel tempo, tale che per una grandezza G legata al fenomeno si abbiano attraverso $\sigma$ discontinuità nelle derivate parziali, a partire da un certo ordine, della funzione mediante la quale si esprime G (in modo che alle due facce di $\sigma$ corrispondono determinazioni di G che soltanto parzialmente si raccordano attraverso $\sigma$ e dalle due parti di $\sigma$ valgono regimi diversi per la grandezza in esame) e le dette discontinuità si propaghino, al muoversi di $\sigma$, con andamento ondoso. Le o. di discontinuità sono dette più particolarmente o. ordinarie di discontinuità quando si debbano distinguere dalle o. d'urto. 
 
\subsection{O. gravitazionali}
O. previste dalla teoria generale della relatività di A. Einstein, emesse da masse in moto, tali che il loro tensore di quadrupolo abbia derivata terza non nulla rispetto al tempo. 

\subsection{O. infrasonore} 
O. longitudinali nell'aria, di frequenza minore di circa 16 Hz. 

\subsection{O di Rayleigh o superficiali}
Sono le onde che si propagano sulla superficie libera dei solidi e hanno interesse nelle applicazioni dell'elettronica e in teoria dei segnali. 

\subsection{O. di Lamb}
Particolari eccitazioni ultrasoniche di tipo elastico di strutture laminari dei solidi; prendono il nome da H. Lamb e sono state utilizzate per l'analisi non distruttiva di materiali e sono quelle che interessano le perturbazioni di tipo sismico. 

\subsection{O. di marea}
Con significato specifico, le o., progressive o stazionarie, che si stabiliscono in un fiume, o in un canale, per effetto delle oscillazioni di marea alla foce. 

\subsection{O. sismiche}
O. elastiche longitudinali e trasversali che si propagano nella Terra dall'ipocentro di un terremoto. Le o. canalizzate si propagano in una stratificazione in cui la velocità di propagazione risulta inferiore sia a quella degli strati sottostanti sia a quella degli strati sovrastanti. Il nome è dovuto al fatto che, a causa delle suddette relazioni di velocità, le o. che tendono a uscire dalla stratificazione vengono rifratte verso l'interno di questa da parte sia degli strati inferiori sia di quelli superiori, per cui la stratificazione agisce come un canale che le convoglia. 

\subsection{O. d'urto}
Nella meccanica dei fluidi, particolare o. di discontinuità, attraverso la quale la componente della velocità del fluido perpendicolare alla superficie d'o. subisce una variazione brusca. Un'o. d'urto può essere schematizzata come in fig. 5. La superficie d'o. $\sigma$ (detta fronte d'urto) separa il fluido in due regioni: la regione a monte, indicata in figura come lato 1, e quella a valle, indicata come lato 2. Per semplicità, si è supposto che il fronte d'urto sia piano e che il fluido incida su di esso perpendicolarmente. 
In generale, non soltanto la velocità v, ma tutte le grandezze che caratterizzano il fluido (la densità $\rho$, la pressione p, la temperatura T, ...) variano in modo brusco attraverso il fronte d'urto. Questo, nella realtà, non è una superficie geometrica, ma uno strato assai sottile (avente uno spessore dell'ordine del cammino libero medio delle molecole), nel quale avvengono processi dissipativi responsabili delle variazioni dei parametri macroscopici del fluido. 
D'altra parte, le discontinuità delle varie grandezze, espresse dai rapporti $\frac{v2}{v1}$, $\frac{\rho_2}{\rho_1}$, $\frac{p2}{p1}$, ... (dove gli indici 1 e 2 si riferiscono alle due regioni del fluido definite in fig. 5), non sono arbitrarie. Occorre, infatti, che i valori di v1, v2, $\rho1$, $\rho2$, p1, p2, ... siano tali da rispettare le leggi di conservazione della massa, dell'energia e della quantità di moto. 
Queste impongono che i flussi di massa, di energia e di quantità di moto che ‘entrano' nella superficie $\sigma$ (dal lato 1) siano uguali a quelli corrispondenti che ne ‘escono' (dal lato 2). Ne deriva che la velocità del fluido diminuisce passando dal lato 1 al lato 2 (v1>v2): più precisamente, si trova che la velocità del fluido deve essere supersonica a monte del fronte d'urto (v1>c1, essendo c1 la velocità del suono sul lato 1) e subsonica a valle di questo (v2<c2, essendo c2 la velocità del suono sul lato 2). Per la conservazione della massa, la disuguaglianza v1>v2 implica l'altra: $\rho1<\rho2$ (cioè il fluido attraversando il fronte d'urto subisce una compressione).

Un'o. d'urto si genera quando un flusso supersonico incontra un ostacolo. Tale situazione si presenta, per es., nel caso di un aereo in volo supersonico (infatti, in un sistema di riferimento solidale con l'aereo, è l'aria a investire questo con velocità supersonica). La fig. 6 illustra schematicamente il fenomeno. Si supponga che l'aereo negli istanti successivi t0, t1, t2, t3, t4, t5 si trovi nelle posizioni P0, P1, P2, P3, P4, P5. In ciascun istante, esso perturba l'aria generando impulsi di onde sonore che si propagano in tutte le direzioni. Tuttavia, poiché l'aereo si muove alla velocità v, maggiore della velocità del suono c, le onde rimangono ‘indietro' rispetto a esso. All'istante t5, i fronti delle onde generate nei punti P0, P1, P2, P3, P4, P5 sono le superfici sferiche tratteggiate in figura, i cui raggi misurano rispettivamente r0=c(t5-t0), r1=c(t5-t1), ...: queste inviluppano una superficie conica, detta cono di Mach, il cui vertice cade nel punto P5. Si trova che le o. sonore interferiscono positivamente (cioè si rafforzano a vicenda) lungo la superficie del cono di Mach, sviluppando, appunto, un'o. d'urto (il cosiddetto boom supersonico). 
Analogo è il caso dell'o. balistica, prodotta da un proiettile in moto supersonico attraverso l'aria. O. d'urto possono anche svilupparsi da o. sonore di ampiezza finita. 
Queste, infatti, si propagano attraverso un fluido, rimanendo inalterate, soltanto se hanno ampiezza infinitesima: se, invece, la loro ampiezza è finita, evolvono progressivamente trasformandosi in o. d'urto. Il fenomeno è illustrato nella fig. 7, che mostra le variazioni spaziali della densità $\rho$ del fluido, intorno al valore medio $\bar{\rho}$, in tre momenti successivi. 
Inizialmente $\rho$ varia sinusoidalmente (fig. 7A). Tuttavia, la regione dove il fluido è compresso ($\rho>\bar{\rho}$) viaggia più velocemente di quella dove esso è rarefatto ($\rho<\bar{\rho}$) . 
Ciò perché la velocità di propagazione delle o. sonore aumenta con la densità: dalle equazioni [19] si ricava infatti 
$c=(k primo \gamma\rho \gamma-1)\frac{1}{2}$. 
Ne segue che la regione di compressione tende a sopravanzare quella di rarefazione che la precede: la variazione di densità diventa pertanto sempre più brusca (fig. 7B) fino a dar luogo a una discontinuità, cioè a un'o. d'urto (fig. 7C). 


 ONDE (fr. ondes; sp. ondas; ted. Wellen; ingl. waves). - È difficile dare una definizione generale di fenomeno ondoso, la quale da un lato abbracci tutti i casi, in cui, sotto qualche aspetto, le onde sono fisicamente rilevabili, e dall'altro sia matematicamente precisa. Quando si confrontano fra loro i varî tipi particolari di fenomeni, che alla nostra intuizione presentano un andamento ondoso, e si cerca di desumerne per astrazione il quid comune, le caratteristiche salienti tendono a sfumare e si finisce col comprendere addirittura tutti i movimenti dinamicamente possibili. Ma, se ci si accontenta d'una indicazione d'orientamento generale, si è condotti ad assumere come caratteristica tipica d'ogni specie di onde (nelle corde o nell'acqua o nell'aria o nel suolo, ecc.) quella che già fu espressa da Leonardo con le parole seguenti: "L'impeto" cioè la propagazione della perturbazione del mezzo o, più in generale, di un qualsiasi elemento saliente "è molto più veloce che ll'acqua, perché molte son le volte che l'onda fuggie il locho della sua creatione, e ll'acqua non si muove di sito, a ssimilitudine delle onde fatte il maggio nelle biade dal corso de venti, che ssi vede correre l'onde per le campagnie, e le biade non si mutano di lor sito" (Del moto e misura dell'acqua, Bologna s. a.). Precisando questa intuizione di Leonardo, si può qualificare ondoso il moto d'un mezzo materiale qualsiasi, quando agli spostamenti delle sue singole particelle si accompagna un moto assai più cospicuo di qualche elemento caratteristico del fenomeno, quale può essere, ad esempio, una propagazione di qualsivoglia perturbazione, o un passaggio d'energia, o, in forma più astratta, uno spostamento della superficie di separazione fra il mezzo di cui si tratta e un altro (in particolare, aria o vuoto) o anche fra due diversi regimi di moto del medesimo mezzo materiale considerato.

È in questo senso vinciano che le onde furono introdotte nella sua teoria della luce da C. Huygens (1690), il quale, riprendendo la concezione aristotelica d'un etere riempiente lo spazio e partecipante ai fenomeni luminosi, ammette che la luce, come il suono nell'aria, consista in un movimento dell'etere "par des surfaces et des ondes sphériques; carje les appelle ondes, à la ressemblance de celles que l'ont voit se former dans l'eau quand on y jette une pierre, qui représentent une telle extension successive en rond, quoique provenant d'une autre cause et seulement dans une surface plane" (Traité de la lumière, ediz. di Parigi 1920, p. 5). Per Huygens è questo un puro modello geometrico; ma esso gli bastò - con l'aggiunta, caso per caso, di particolari ipotesi sussidiarie - a rendere conto, in modo soddisfacente se non perfetto, dei fenomeni della propagazione rettilinea, della riflessione, della rifrazione e della doppia rifrazione, da lui stesso scoperta, dello spato d'Islanda.

È ben noto che per tutto il sec. XVIII sulla teoria ondulatoria di Huygens prevalse quella emissiva o corpuscolare di Newton, fino a quando, sul principio del secolo successivo, T. Young e A. Fresnel, a includere in uno schema ondulatorio nuovi ordini di fenomeni osservati, aggiunsero alla veduta puramente geometrica di Huygens il supporto d'una teoria dinamica. D'altro canto va ricordato che la stessa ottica geometrica di Huygens trovò più tardi, per opera di W.R. Hamilton e di S. Lie, il suo sviluppo e il suo coronamento nelle trasformazioni di contatto, collegate alle cosiddette equazioni canoniche della dinamica (cfr., per es., T. Levi-Civita e U. Amaldi, Lezioni di meccanica razionale, II, 11, Bologna 1927, cap. X).

Per la storia del dualismo fra le concezioni corpuscolari e quelle ondulatorie, che, con vicenda alterna, hanno dominato, pressoché fin dalle origini, l'evoluzione delle teorie fisiche, si vedano le voci fisica; luce; quantistica, meccanica. La prima parte del presente articolo (nn.1-16) è dedicata a un esame sommario dei varî aspetti, sotto cui la nozione di onda si schematizza matematicamente, in relazione ai diversi ordini di fenomeni, nei quali essa si presenta come elemento essenziale; e si concluderà con l'indicazione d'un modello astratto, in cui si può rispecchiare l'odierna tendenza dell'accennato dualismo a comporsi in una veduta unitaria più elevata. Nella seconda parte (nn. 17-31) verrà dato un rapido sguardo a codesti varî ordini di fenomeni fisici di carattere ondoso.

Schematizzazione matematica dei fenomeni ondosi.

1. Corde vibranti. - Per fissare le idee, si consideri il fenomeno delle vibrazioni trasversali d'una corda. Più precisamente si prenda in esame il caso tipico d'un tratto di corda omogenea, flessibile e sensibilmente inestendibile, di lunghezza l, il quale sia inizialmente disposto secondo il segmento rettilineo OA; e poi, per l'intervento di opportune sollecitazioni, vibri in modo che i singoli suoi elementi materiali compiano piccole oscillazioni in un ben determinato piano passante per la retta OA e in direzione ortogonale ad essa. Ove si denoti con s l'ascissa d'un generico elemento materiale della corda, nella sua configurazione rettilinea iniziale, e con t il tempo, lo spostamento trasversale, che istante per istante subisce l'elemento considerato, sarà una certa funzione $\eta$ (s, t) di s e t; e dalla teoria delle piccole oscillazioni dei fili flessibili e inestendibili (d'Alembert; v. fili e verghe) risulta che una tale funzione $\eta$ (s, t) deve soddisfare l'equazione a derivate parziali del 2° ordine

dove le costanti $\rho$ e T rappresentano rispettivamente la densità lineare della corda (massa per unità di lunghezza) e il valore della tensione nello stato iniziale. Lo spostamento $\eta$, oltre questa equazione indefinita (cioè valida ad ogni istante per ogni elemento della corda), deve, caso per caso, soddisfare, in dipendenza dal collegamento della corda con altri corpi, convenienti condizioni ai limiti (cioè relative agli estremi O e A), le quali, nel caso particolare che gli estremi siano fissi, si riducono alla condizione che in O e A lo spostamento sia sempre nullo, vale a dire si abbia, per ogni possibile valore del tempo t, $\eta$ (0, t) = $\eta$ (l, t) = 0. Infine, per la natura stessa del fenomeno fisic0, la massima ampiezza dello spostamento $\eta$ d'ogni singola particella deve risultare piccola di fronte alla lunghezza della corda; e questa essenziale circostanza di fatto s'introduce nella trattazione matematica del problema mediante l'ipotesi che il rapporto $\eta$/l si possa considerare come una quantità del 1° ordine, vale a dire tale che se ne possano trascurare tutte le potenze dalla seconda in su. Di questa ipotesi si tiene conto, supponendo - come è lecito senza contraddire all'equazione indefinita (1) e ai tipi più comuni di condizioni ai limiti - che la funzione contenga come fattore una costante $\epsilon$ piccola a piacere, che ne fissi l'ordine di grandezza.

Nel seguito, ad evitare sviluppi non consentiti dai limiti imposti a questo articolo, si studieranno esclusivamente le conseguenze, che derivano dalla validità dell'equazione indefinita (1), ammettendo una volta per tutte di considerarne soltanto quelle soluzioni, le quali rendano soddisfatte le condizioni ai limiti, che, caso per caso, potranno essere imposte dal problema trattato. Perciò tutto quanto verrà qui detto varrà incondizionatamente (cioè per ogni possibile coppia di valori di s e t) nel caso di una corda indefinita, per la quale non intervenga alcuna condizione ai limiti; mentre varrà soltanto per intervalli convenientemente delimitati tutte le volte che si debba tenere conto di circostanze fisiche, schematizzabili in condizioni ai limiti, come accade, già nel caso delle corde, per gli effetti di riflessione agli estremi, e, più in generale, per tutti i fenomeni di riflessione o di rifrazione nell'ottica o in qualsiasi altro campo.

2. Onde progressive permanenti. - Per precisare gli aspetti ondulatorî, che si presentano nelle vibrazioni delle corde, si ricordi che la più generale soluzione dell'equazione indefinita (1) è della forma (v. equazioni, n. 4; oscillazioni e vibrazioni)

dove $\eta$1 e $\eta$2 denotano due funzioni arbitrarie dei rispettivi argomenti s - Vt, s + Vt, mentre V è una costante legata ai parametri fisici $\rho$ e T dalla relazione

e si fissi l'attenzione su una soluzione della forma (corrispondente all'ipotesi che la $\eta$2 sia identicamente nulla)

avvertendo che, in accordo con l'impostazione del n. prec., si dovrà intendere che la funzione $\eta$1 contenga come fattore una costante $\epsilon$, piccola a piacere. In questo caso le due variabili s e t (elemento della corda e tempo) non influiscono sul fenomeno l'una indipendentemente dall'altra, ma soltanto attraverso la loro combinazione s - Vt, talché, se esse variano in guisa che questo binomio si mantenga costante, si conserva inalterato anche il corrispondente spostamento. Ora la condizione

non è altro che l'equazione oraria d'un moto uniforme, lungo la retta OA, di velocità V; onde si conclude che, durante la vibrazione della corda, governata dall'equazione (4), un qualsiasi osservatore, che si sposti lungo OA con la velocità costante V, vede ad ogni istante determinarsi davanti a sé un medesimo spostamento $\eta$. In altre parole, tutto va come se la configurazione assunta dalla corda in un dato istante si spostasse, mantenendosi inalterata, da O verso A con la velocità V. Se poi si tiene conto che questa velocità V =

è del tutto indipendente dalla velocità trasversale propria di ciascuna particella, la quale è data da $\partial$$\eta$/$\partial$t ed è quindi dello stesso ordine di grandezza di $\eta$, cioè di quel suo fattore $\epsilon$, che va ritenuto piccolissimo, si riconosce che si è di fronte a un tipico fenomeno ondoso. Precisamente le onde di questo tipo si dicono, per ragioni ovvie, progressive permanenti; il binomio s1 = s - Vt, da cui dipende esclusivamente il fenomeno, si dice fase, e la velocità V, non legata ad alcun elemento materiale, bensì dipendente dai parametri fisici $\rho$ e T traverso la (3), si chiama velocità di propagazione dell'onda (e talvolta anche velocità di fase).

È poi manifesto che anche una soluzione della (1), la quale abbia la forma, per così dire complementare

definisce un'onda permanente; e quest'onda si differenzia da quella corrispondente alla soluzione (4) soltanto per il senso della propagazione, in quanto la rispettiva velocità è data da - V anziché da V (onda retrograda).

3. Onde progressive permanenti in più dimensioni. - Anche nel caso di fenomeni in tre dimensioni, sonori o elastici o elettromagnetici, si è pervenuti alla determinazione di onde, cercando quelle classi particolari di soluzioni (dei sistemi differenziali, o alle derivate parziali, definenti i fenomeni stessi), che dipendono da un unico argomento funzione lineare delle tre coordinate di spazio x1, x2, x3 e del tempo t, cioè del tipo

dove le c sono costanti da determinarsi in modo opportuno. Ammesso che si tratti soluzioni, le quali dipendano in qualche modo anche dal posto (cioè non siano funzioni unicamente di t), si dovrà ritenere diversa da zero almeno una delle c1, c2, c3, ossia non nullo il vettore c, avente c1, c2, c3 per componenti secondo gli assi coordinati. Con ciò si può anche considerare come unico argomento, da cui dipendono le soluzioni in questione, l'espressione s1 = $\sigma$/c, in cui c denota la lunghezza del vettore c; e se si osserva che i rapporti $\alpha$i =-ci/c (i = 1, 2, 3) sono coseni direttori (del vettore c) e si pone V = - c0/c, l'argomento s1 assume anche qui l'aspetto s1 = s - Vt, dove s = $\alpha$1 x1 + $\alpha$2 x2 + $\alpha$3 x3 è un'ascissa nella direzione ($\alpha$1, $\alpha$2, $\alpha$3). Si tratta dunque di onde piane, nel senso che lo stato vibratorio dipende, ad ogni istante t, soltanto da s, ed è quindi identico in tutti i punti di un medesimo piano s = cost.; e anche qui, come nel caso delle corde, il fenomeno ha carattere stazionario rispetto ad ogni osservatore, che si muova nella direzione ($\alpha$1, $\alpha$2, $\alpha$3), in modo da soddisfare la condizione s1 = s - Vt = cost., cioè con la velocità costante V.

Un'impostazione più generale si ha, assumendo s funzione qualsiasi (anziché lineare) di x1, x2 x3 e supponendo che i parametri determinativi del fenomeno dipendano non soltanto da s1 = s - Vt, ma anche da un altro argomento puramente spaziale. In quest'ultimo tipo rientrano le cosiddette onde sferiche.

E tipi ancora più generali di onde, concepite in tal modo, furono studiati, sotto punti di vista diversi, da H. Bateman (in Electrical and optical wove motion, Cambridge 1915) e da G. A. Maggi (in Rend. Lincei, s. 5, XXIX, 1920, pp. 371-378)

4. Pacchetto d'onde. - Nella schematizzazione di notevoli fenomeni fisici interviene il caso di quelle onde permanenti, per cui lo spostamento $\eta$ è (rigorosamente o, almeno, sensibilmente) diverso da zero soltanto in un piccolo intorno, di ampiezza 2$\delta$, d'un valore determinato della fase, p. es. di s1 = 0. Di qui consegue che in un generico istante t il tratto, nel quale la vibrazione k sensibile, risulta compreso fra s = Vt - $\delta$ e s = Vt + $\delta$, talché il suo centro si sposta rettilineamente, nel mezzo vibrante, con velocità V. È ciò che, in una dimensione, si chiama pacchetto d'onde, e ha il suo analogo anche in due o tre dimensioni

5. Onde permanenti sinusoidali o armoniche. - Questo tipo importante di onde permanenti si ha, supponendo $\eta$ d'una delle due forme

dove $\lambda$, $\alpha$, $\beta$ denotano costanti, a priori indeterminate. Ma queste due espressioni di $\eta$ sono riducibili l'una all'altra con la posizione $\beta$ = $\alpha$ - $\pi$/2; e d'altra parte in ciascuna di esse la costante $\alpha$ o $\beta$ si può ridurre a zero Lon un cambiamento d'origine degli spazî o dei tempi; onde bas'terà considerare il caso

Ponendo per brevità T = $\lambda$V-1, ossia

e

si dà alla (5) l'aspetto più semplice

Dal fatto, che sen x è, rispetto a x, funzione periodica di periodo 2h, segue che la (5) è funzione periodica della fase si, col periodo1, e, quindi, tanto di s quanto di t, con i periodi T e $\lambda$ rispettivamente. Ciò vuol dire che, se si fissa l'attenzione su un determinato posto (cioè si attribuisce un valore ben determinato a s, ponendo, per es., s = 0), si ha per la vibrazione trasversale della corrispondente particella un'equazione oraria sinusoidale, la quale ammette il periodo T, che si chiama il periodo del moto ondoso, di cui si tratta (fig.1). Se, invece, si fissa l'attenzione su un determinato istante t (per es., t = 0), la legge con cui si susseguono nel mezzo vibrante, in quell'istante, gli spostamenti trasversali delle varie particelle, è anch'essa sinusoidale, col periodo $\lambda$, che si chiama lunghezza d'onda, in quanto a intervalli $\lambda$ gli spostamenti si riproducono identicamente (fig. 2). I due periodi T e $\lambda$ (aventi le dimensioni d'un tempo e d'una lunghezza) sono legati fra loro dalla relazione (6), che ricorda la legge oraria d'un moto uniforme. Con essi si considerano anche i rispettivi reciproci

cui si dànno rispettivamente i nomi di frequenza e di ondulanza, e che permettono di dare all'equazione (5) delle onde sinusoidali la forma

Dalla natura sinusoidale dell'equazione oraria delle vibrazioni trasversali d'una generica particella risulta che il corrispondente spostamento, nella sua variazione periodica, tocca alternativamente, a intervalli di tempo di durata T/2, massimi di valore $\epsilon$ e minimi di valori - $\epsilon$, che si designano col nome comune di ventri d'onda; e a metà dell'intervallo di tempo T/2, che trascorre tra la formazione d'un ventre e quella del successivo, s'incontra un nodo, cioè un istante in cui lo spostamento si annulla, sicché la particella passa per la sua posizione naturale; è dunque T/4 il tempo occorrente, perché la particella passi da un ventre al nodo successivo, o viceversa. Analogo è l'andamento del diagramma degli spostamenti delle varie particelle in un medesimo istante, con la sola differenza, che qui la distanza (non più temporale, ma spaziale) fra un ventre e il nodo successivo, o viceversa, è data da $\lambda$/4. Perciò, nel caso tipico della corda vibrante, si può dare un'immagine plastica del fenomeno, dicendo che, davanti a un osservatore fisso in un punto della corda, si formano alternativamente ventri e nodi a intervalli di tempo T/4; e, quando gli accade di vedere, dinnanzi a sé, un ventre o un nodo, egli vede, nel medesimo istante, formarsi, un tratto $\lambda$/4 più avanti e più indietro, un nodo o, rispettivamente, un ventre.

6. Onde permanenti periodiche generali. - Le stesse considerazioni qualitative valgono ovviamente anche nel caso, in cui lo spostamento $\eta$, anziché sinusoidale, sia una qualsiasi funzione periodica di periodo 1 dell'unico argomento

Esempî particolarmente semplici si ottengono, riprendendo l'onda sinusoidale (5) e rimettendo in evidenza nell'argomento quella costante additiva, di cui ci siamo liberati al principio del n. prec., cioè, come si suol dire, sfasando un'onda sinusoidale. Si ha così

dove s0 denota una costante, avente, al pari di s1, le dimensioni d'una lunghezza. Si parla qui di sfasamento della soluzione sinusoidale tipica, perché tutto si riduce a spostare di s0 l'origine degli spazî, il che equivale a uno scorrimento di ampiezza s0 (in valore e segno) del diagramma della fig. 2. In particolare, se s0 corrisponde a un anticipo o a un ritardo di 1/4 di periodo, si passa dal diagramma del seno a quello del coseno oppure del coseno cambiato di segno.

D'altra parte va ricordato che ogni funzione periodica $\eta$ (s1) si può rappresentare come somma o, al limite, come serie di seni e coseni dei multipli della variabile indipendente, ottenendosi (v. fourier: Serie del Fourier), se il periodo di $\eta$ (s1) è 1,

o

dove le a0, an, bn o le a0 cn, $\sigma$n sono convenienti costanti. In questo modo, ove si prescinda dal termine costante a0/2 (che corrisponde a una traslazione nel senso dello spostamento, in generale incompatibile con le condizioni ai limiti), l'onda periodica risulta generata dalla sovrapposizione di onde sinusoidali, aventi ordinatamente le lunghezze d'onda $\lambda$, $\lambda$/2, $\lambda$/3, $\lambda$4, ... e i periodi T, T/2, T/3, T/4, ... Tutto ciò, naturalmente, vale incondizionatamente nel caso d' una corda vibrante indefinita; quando, invece, si tratti d'una corda di data lunghezza l, intervengono, come già si è notato, le condizioni ai limiti, che possono essere di tipi svariati e influire in modo essenziale sull'andamento del fenomeno (dando luogo, ad esempio, a riflessioni parziali o totali agli estremi, supposti fissi, o a particolari condizioni dinamiche, se un estremo è libero). Chi voglia, anche solo da un punto di vista elementare, rendersi conto di queste circostanze, che qui di proposito si trascurano, ma che nei casi concreti sono talvolta particolarmente importanti (fenomeni elastici e, in particolare, acustici), può consultare qualcuno dei trattati, indicati nella bibliografia.

7. Principio di sovrapposizione. - Quando un fenomeno è retto da equazioni differenziali, o a derivate parziali, lineari omogenee, quale è appunto la (1), la somma di due soluzioni è ancora una soluzione; e in ciò consiste il cosiddetto "principio di sovrapposizione", di cui un caso particolare è già stato incontrato al n. prec. Ma, restando dapprima nella massima generalità, va osservato che, quando si tiene conto di tutte le caratteristiche d'un dato fenomeno, e, in particolare, non soltanto delle equazioni indefinite, bensì anche di condizioni ai limiti, accade, in generale, che la somma di due soluzioni soddisfacenti queste condizioni non è più tale, sicché il principio di sovrapposizione va applicato con le dovute cautele. Se poi si considerano, più in particolare, problemi relativi alle onde, bisogna notare che, anche nel caso favorevole, in cui, per il fenomeno di cui si tratta, la somma di due soluzioni, corrispondenti ciascuna ad una propagazione di onde, sia ancora una soluzione, succede generalmente che il carattere ondoso, che si presenta in ciascuna di codeste due soluzioni, non si ritrovi più nella somma; e un esempio particolarmente espressivo di questa possibilità è dato dall'integrale generale della (1)

i cui addendi rappresentano due treni di onde, le prime progressive, le seconde retrograde, mentre la somma, come atta a rappresentare il più generale moto vibratorio della corda, non possiede affatto il carattere, che contraddistingue i moti ondosi.

8. Onde sistatiche, dette anche stazionarie. - Uno stato vibratorio, che merita particolare considerazione, si ha in corrispondenza di quelle soluzioni della (1), che si presentano come prodotti di due funzioni, l'una della sola t, l'altra della sola s,

In tal caso i valori di s, che annullano la f(s), annullano la $\eta$, qualunque sia t, sicché il fenomeno vibratorio è a nodi fissi. Inoltre, avendosi

risulta manifesto che i valori di s, per cui si annulla la f' (s), annullano, per qualsiasi t, anche la $\partial$$\eta$/$\partial$s. Sono perciò fissi gli eventuali massimi e minimi dello spostamento $\eta$ (v. massimi e minimi), cioè anche i ventri. Un caso semplice, in cui si trovano realizzate queste circostanze, si ha componendo due onde sinusoidali, aventi eguali l'ampiezza, la fase, la lunghezza e il periodo, ma l'una progressiva e l'altra retrograda, cioè ponendo

Questo stato vibratorio si trova spesso designato come esempio tipico di "onde stazionarie", ma in verità né il sostantivo, né l'aggettivo sono appropriati, perché anzitutto il moto (8), risultante dalla sovrapposizione di due treni d'onde propagantisi in senso opposto, non ha carattere di moto ondoso, secondo il criterio generale indicato da principio; e in secondo luogo la qualifica di "stazionario" ha correntemente in meccanica un senso diverso, cioè viene usato come sinonimo di "permanente"; talché, ad evitare ambiguità, gioverebbe adottare un aggettivo diverso, come già usano gl'Inglesi e i Tedeschi, che chiamano questo tipo di fenomeno vibratorio "standing", e, rispettivamente "stehend", in contrapposto a "steady" e "stationär" che rispondono al nostro usuale stazionario. Perciò i fenomeni vibratorî (8) furono qui chiamati sistotici.

9. Gruppo di onde periodiche e velocità di gruppo. - Nel caso schematico di una dimensione (p. es., corda vibrante) si chiama, ovviamente, "gruppo d'onde" ogni stato vibratorio determinato dalla sovrapposizione di più onde permanenti del tipo considerato al n. 2, e quindi corrispondente a uno spostamento della forma

con

dove le $\eta$i sono altrettante funzioni periodiche di periodo 1 del rispettivo argomento si, = ki s - $\nu$i t; e, in generale, le ondulanze ki, e le frequenze $\nu$i variano con l'indice i. Come già si è notato nel caso di due sole onde componenti (n. 7), il fenomeno non ha, in generale, carattere ondoso; ma, se si fissa più particolarmente l'attenzione sul caso n = 2, cioè si considera un gruppo binario, che, per semplicità, scriveremo

si può definire una certa velocità costante U, detta ielocità di gruppo, che è tale che, rispetto a un osservatore mobile con siffatta velocità nella direzione di propagazione, il fenomeno, senza essere propriamente ondoso, risulti periodico rispetto al tempo. Per dimostrarlo, si consideri dapprima un osservatore $\Omega$, mobile con una velocità costante U generica; e, supposto che nell'istante t = 0 esso occupi la posizione s = 0, si indichi con $\xi$ l'ascissa rispetto a $\Omega$ della particella generica del sistema vibrante, che ha l'ascissa assoluta s. Si ha, istante per istante, s = $\xi$ + Ut; e, se in $\eta$1, $\eta$2, si fa apparire, in luogo di s, l'ascissa relativa $\xi$, i rispettivi argomenti diventano

Poiché ciascuna delle due funzioni $\eta$1, $\eta$2 è, rispetto al proprio argomento, periodica di periodo 1, è manifesto che, se si sceglie U in modo che siano eguali i coefficienti di t nei due argomenti (9), le due funzioni risultano periodiche rispetto a t, con uno stesso periodo, eguale al reciproco del valore comune dei due coefficienti accennati. Si è così condotti a porre $\nu$ - kU = $\nu$' - k'U; onde risulta

ed è questa l'espressione della "velocità di gruppo".

Una tale denominazione appare ben giustificata, se si considera il caso in cui, essendo sinusoidali entrambi i treni d'onde componenti, le costanti k', $\nu$' sono prossime a k, $\nu$ rispettivamente. Si ha in tal caso

ossia, introducendo l'ascissa $\xi$ relativa all'osservatore $\Omega$, animato di velocità U, e tenendo conto della espressione (10) di questa velocità,

Per riconoscere la caratteristica saliente di questo fenomeno vibratorio rispetto all'osservatore mobile $\Omega$, si avverta che in ogni punto Pn di ascissa relativa

dove n L un intero (positivo o negativo) qualsiasi, si ha rigorosamente $\eta$ = 0, e nella prossimità di ciascuno di questi punti lo spostamento $\eta$ è molto piccolo, sicché si tratta d'una regione di sensibile quiete (così detta "regione di acqua morta", se queste generalità si applicano alle onde di canale, di cui si parlerà tra poco). Queste regioni di tranquillità sono separate da intervalli di ampiezza costante 1/4|k - k'|, che, per essere k' prossimo a k, è molto più grande delle lunghezze d'onda, sensibilmente coincidenti, 1/k, 1/k'. dei due treni componenti. Nel tratto compreso tra due regioni di quiete consecutive l'andamento di $\eta$, che, in quanto dipende da due distinti argomenti lineari in $\xi$ e t, non è ondoso, dà luogo a un diagramma serpeggiante in un modo qualsiasi e variabile nel tempo, che si direbbe "increspato", se si trattasse d'uno specchio d'acqua. Si vede così che la velocità di gruppo è quella, con cui si spostano, rispetto ad un osservatore fisso, questi successivi tratti increspati, solidalmente con le regioni interposte di tranquillità.

10. Fenomeni dispersivi. - Nelle precedenti generalità di carattere cinematico non si è tenuto conto di eventuali relazioni fra lunghezza d'onda e periodo o, ciò che è lo stesso, fra ondulanza e frequenza. Quando le due costanti d'una di queste coppie non siano fra loro indipendenti, bensì funzioni l'una dell'altra il fenomeno si dice dispersivo, in quanto, a norma della (6), la velocità di propagazione viene a dipendere dalla frequenza. In tal caso la formula (10), che dà la velocità di gruppo, acquista un significato particolarmente interessante. Assunta, come argomento indipendente, l'ondulanza k, si possono esprimere in funzione di essa tanto la frequenza $\nu$ = 1/T, quanto la velocità di propagazione V = $\lambda$/T = $\nu$/k. Nell'ipotesi particolare di due treni d'onde progressive periodiche, aventi ondulanze e frecluenze prossime, si può assimilare k' - k a un differenziale dk e $\nu$' - $\nu$ al differenziale d$\nu$ della corrispondente funzione, sicché, in virtù della (10), si ha per la velocità di gruppo l'espressione

e di qui, tenendo conto che si ha $\nu$ = kV, dove V è la velocità di fase, cioè la velocità di propagazione del primo treno d'onde e, quindi, sensibilmente anche del secondo, si deduce

Quando si tratti di fenomeni luminosi, giova introdurre nel secondo termine di quest'ultima espressione di U, in luogo di V, l'indice di rifrazione n = c/V, dove c denota la velocità della luce nel vuoto; e si perviene alla formula del Raylegh

da cui, in quanto, in generale, l'indice di rifrazione, al crescere dell:i lunghezza d'onda, decresce (dispersione normale) - talché si ha dn/d$\lambda$ ⟨ 0 - si deduce che, in questo caso, la velocità di gruppo è più piccola della velocità di fase.

11. Velocità di propagazione in alcuni casi importanti. - Analogamente a quanto si è visto nel caso delle onde trasversali d'una corda vibrante (n. 2), anche in ogni altro tipo di fenomeni fisici, quando si cercano per le corrispondenti equazioni differenziali le soluzioni aventi carattere ondoso, si è condotti ad esprimere la velocità di propagazione per mezzo delle costanti fisiche salienti del fenomeno. Così, tornando, in primo luogo, al caso d'una corda, ma supponendola elastica e considerandone le vibrazioni longitudinali, cioè non più normali, bensì parallele alla direzione di propagazione, si trova

dove $\rho$ è ancora la densità lineare, mentre E denota il cosiddetto modulo di elasticità longitudinale o dello Young, cioè il rapporto tra una trazione longitudinale, a cui si sottoponga la corda, e il corrispondente allungamento unitario.

Per le onde sonore nell'aria, considerata come un gas perfetto in regime adiabatico, si ha

dove $\gamma$ = 1,41 è il rapporto fra i due calori specifici dell'aria, a pressione e a volume costante, e p, $\rho$ rappresentano la pressione e la densità dell'aria nello stato non perturbato da vibrazioni sonore. Com'è ben noto, V in condizioni normali risulta di circa 330 m./sec.

Nel caso di onde elastiche in un mezzo isotropo si trova, indicando con $\lambda$ e $\mu$ le cosiddette costanti elastiche del Lamé,

secondo che si tratta di onde longitudinali o trasversali. Poiché le costanti $\lambda$ e $\mu$ sono positive, si vede che la velocità delle onde longitudinali è sempre maggiore di quella delle onde trasversali; e va osservato che entrambe queste velocità risultano più rilevanti di quella del suono nell'aria. Siccome poi $\lambda$ e $\mu$, pur essendo in generale notevolmente diverse, hanno per altro un medesimo ordine di grandezza, il rapporto della prima alla seconda velocità non può essere di molto inferiore a 2: ad es., nell'acciaio le due velocità sono, rispettivamente, di 6100 e 3200 m./sec. Non è inutile aggiungere che per i corpi isotropi il modulo dello Young è legato alle costanti del Lamé dalla relazione

In un mezzo elastico, che presenti una superficie libera, come accade ad es. in un semispazio (suolo elastico), si possono propagare onde, che si smorzino rapidamente dalla superficie libera verso l'interno, dette onde superficiali. Considerate per la prima volta dal Raylegh, sono state studiate o generalizzate, con particolare riguardo alle applicazioni sismiche, da L. De Marchi, C. Somigliana, R. Einaudi (in Rend. Lincei, s. 5ª, XXV, 1916; XXVI, 1917; XXVII, 1918; s. 6ª, XIX, 1934). La velocità di propagazione di queste onde è all'incirca eguale ai 95/100 di quella delle onde trasversal i.

Infine in un dielettrico omogeneo e isotropo, di cui siano $\epsilon$ e $\mu$ la costante dielettrica e la permeabilità magnetica, la velocità di propagazione delle onde elettromagnetiche (e, in particolare, luminose) è data dalla celebre formula del Maxwell

dove c rappresenta al solito la velocità della luce nel vuoto (che, come si vede, rientra fra codeste velocità di onde elettromagnetiche per $\epsilon$ = $\mu$ = 1). Se poi il dielettrico è anisotropo, pur conservando l'isotropia magnetica, sono ancora possibili onde piane in tutte le direzioni, ma varia dall'una all'altra di queste la velocità di propagazione. Nei cosiddetti mezzi biassici, ove si denotino con V1, V2, V3 le velocità di propagazione delle onde normali agli assi ottici, la velocità V, con cui si possono propagare le onde normali a una generica direzione di coseni $\alpha$1, $\alpha$2, $\alpha$3, è definita dall'equazione

e di qui si ricava per una tale velocità un'espressiva costruzione geometrica, ricorrendo alla superficie del Fresnel, cioè alla superficie algebrica di 40 ordine (e 4ª classe), che rispetto ai medesimi assi ottici ammette l'equazione

dove r2 = x12 + x22 + x32.

12. Onde nei liquidi. Onde di canale. - Si consideri un canale rettilineo a sezione rettangolare, di cui il fondo sia orizzontale e le due pareti siano verticali; e si supponga che il liquido contenuto nel canale - che diremo senz'altro acqua - avvenga parallelamente alle sponde e abbia il medesimo andamento in tutte le sezioni longitudinali, cioè in tutti i piani verticali paralleli alle sponde. Già il Lagrange, introducendo l'ipotesi semplificatrice che sia trascurabile di fronte all'accelerazione g della gravità l'accelerazione verticale delle singole particelle liquide, considerò le onde progressive di tipo permanente, cioè quei moti dell'acqua, nei quali il pelo libero l si sposta senza alterazione di forma, progredendo con una velocità costante V, mentre le singole particelle, anziché essere animate da una velocità comparabile con V, compiono soltanto piccole oscillazioni con velocità locale di valore medio nullo.

Ora, in relazione alla forma del pelo libero, i tipi più studiati si riducono alle onde periodiche (dette dai Francesi "la houle") e all'onda solitaria. Come ben si comprende, nel primo caso la linea l consta di tanti archi consecutivi eguali, riproducentisi periodicamente a intervalli costanti $\lambda$ (lunghezza d'onda); e l'andamento qualitativo è quello di una curva sinusoidale (fig. 3). Invece l'onda solitaria, quale si può provocare immettendo di colpo, con opportune modalità, una massa d'acqua in un canale inizialmente in quiete, consta di un'unica intumescenza, simmetrica rispetto all'ordinata massima (fig. 4). Essa fu studiata sperimentalmente da Scott Russell (1844) e teoricamente, in modo del tutto indipendente, da J. Boussinesq (1871) e da lord Raylegh (1876).

Un tipo rigoroso di onde periodiche è dato dalle cosiddette onde trocoidali trovate, indipendentemente, da F. J. von Gerstner e da W. J. M. Rankine. In esse il pelo libero ha appunto forma di trocoide (o cicloide accorciata; v. cicloide), e le singole particelle fluide descrivono piccole circonferenze, che via via vanno rimpicciolendo verso il fondo. La velocità di propagazione delle onde è data in questo caso da

dove, al solito, g denota l'accelerazione di gravità e $\lambda$ la lunghezza d'onda. Questa soluzione di von Gerstner, per l'espressiva semplicità delle sue caratteristiche, è largamente usata nelle applicazioni idrauliche e nautiche, come base teorica di primo approccio, per apprezzare in qualche modo le molteplici influenze perturbatrici, che si presentano nella pratica; ma essa presenta un inconveniente, per cui l'ufficio conferitole di schema teorico appare giustificato soltanto in via provvisoria, finché non si trovi qualcosa di meglio; e questo inconveniente consiste nel carattere vorticoso o rotazionale delle corrispondenti vibrazioni delle particelle liquide, mentre in un liquido perfetto, sotto l'azione di forze conservative, si possono destare soltanto moti irrotazionali (v. idrodinamica, n. 6).

Conviene dunque cercare altre onde progressive di tipo permanente, che, pur avvicinandosi, per quanto è possibile, alla semplicità di quelle di von Gerstner, siano dovute a vibrazioni irrotazionali delle particelle liquide. Tali sono già le onde poc'anzi accennate del Lagrange, le quali hanno effettivamente una particolare importanza nello studio delle maree; ma mal risponde alla realtà fisica l'ipotesi restrittiva, cui è subordinata la loro esistenza (trascurabilità, di fronte a g, dell'accelerazione verticale delle particelle fluide). Un tipo elementare di onde periodiche, permanenti e irrotazionali, che meglio si presta alle applicazioni, è quello delle onde semplici o sinusoidali di G. B. Airy. Anch'esse sono approssimate, ma l'approssimazione in questo caso è rispondente alla natura fisica del fenomeno, in quanto si riduce ad ammettere che si possa trattare come quantità del 1° ordine il rapporto fra la massima velocità delle particelle e la velocità di V di propagazione. Questa V è definita dalla formula dell'Airy

dove, attribuendo a g e $\lambda$ il solito significato e denotando con h la profondità del canale, si è posto

e tangh $\alpha$ denota la "tangente iperbolica di $\alpha$", cioè il quoziente senh $\alpha$: cosh $\alpha$ (v. funzione, n. 42). Per le cosiddette onde lunqhe, cioè aventi una lunghezza $\lambda$ molto grande rispetto alla profondità h (e tali sono normalmente le onde di marea), $\alpha$ è piccolo e tangh $\alpha$/$\alpha$ si può sensibilmente confondere con l'unità, talché si ottiene per la velocità V l'espressione

che già era stata trovata dal Lagrange nella sua approssimazione. A questa stessa espressione va ravvicinata quella della velocità di propagazione dell'onda solitaria, che è data, ove con a si denoti l'altezza dell'onda (quota della sommità dell'intumescenza sul livello imperturbato), da

e l'analogia delle due espressioni appare in un certo senso giustificata, in quanto l'onda solitaria si può considerare come un caso limite d'una onda periodica, quando la lunghezza $\lambda$ tenda all'infinito. Se, invece, si tratta di onde brevi (e praticamente si possono ritenere tali tutte quelle, per cui la lunghezza non superi il doppio della profondità del canale) si può confondere con l'unità la tangh $\alpha$, onde risulta

cioè si ritrova l'espressione valida per la velocità di propagazione delle onde trocoidali; e anche questo risultato non può meravigliare, se si tiene conto che le onde del von Gerstner, pur essendo rotazionali, vanno in ogni caso considerate come brevi, in quanto si riferiscono a una profondità infinita.

Per le onde semplici dell'Airy, che, come si è detto, corrispondono a una soluzione di prima approssimazione, è valido il principio di sovrapposizione (n. 7). In particolare, si può considerare quel moto dell'acqua, che risulta dalla sovrapposizione di due onde semplici, aventi comuni la lunghezza, il periodo e l'ampiezza, ma dirette in senso opposto, e che, come si è visto al n. 8, dà luogo a onde sistatiche, cioè aventi nodi e ventri fissi. È di questo tipo la cosiddetta maretta (clapotis" dei Francesi).

Nelle onde di canale, oltre la proprietà generale che è poco rilevante il moto assoluto delle particelle liquide di fronte alla velocità di propagazione, si verifica l'importante fatto meccanico che negli strati profondi non vi è, in media, alcun trasporto globale di liquido: ove la propagazione ondosa dia luogo a un qualche trasporto di materia, questo è esclusivamente dovuto agli strati superficiali. Così nel caso tipico delle onde semplici dell'Airy, le cose vanno come se vi fosse, in media, un trasporto di liquido, la cui velocità $\gamma$ (molto piccola di fronte a V) è data da

col solito significato di g, h, a; e di qui risulta che l'intensità del trasporto varia in ragione diretta del quadrato dell'altezza dell'onda e in ragione inversa della velocità di propagazione e della profondità del canale.

13. Soluzione rigorosa del problema. - Di fronte a risultati così espressivi, ma approssimati, era naturale porre il problema matematico della determinazione rigorosa di onde periodiche, permanenti e irrotazionali. G. G. Stokes e lord Raylegh si erano spinti, su questa via, ad approssimazioni ulteriori; e l'esistenza rigorosa d'una soluzione esatta fu dimostrata, nel caso di canali molto profondi, da T. Levi-Civita (1925), il quale assegnò anche le formule adatte al calcolo effettivo, trovando per la velocità di propagazione l'espressione

dove, essendo a la massima altezza dell'onda sul livello medio del canale, si è posto

Quando a sia trascurabile di fronte alla lunghezza d'onda $\lambda$, si ricade, com'è naturale, sulle onde brevi dell'Airy.

In seguito alla ricerca del Levi-Civita, l'indagine è stata estesa da D. J. Struyk (1926) al caso di canali di profondità non più grandissima, ma ben determinata, mentre A. Weinstein (1926) ha spinto ulteriormente l'approssimazione di Boussinesq-Raylegh per l'onda solitaria.

14. Problemi varî di moto ondoso. - I limiti imposti a questo articolo non ci permettono di dilungarci su altri tipi, pur notevoli, di onde, né sulle loro caratteristiche più complesse, né sui problemi fisici di cui esse forniscono la schematizzazione. Perciò ci limitiamo ad una semplice enumerazione: in una dimensione, le onde che si producono alla superficie di separazione di due liquidi di diversa densità (J.-M. Burgers,.N. Kotchine), e l'influenza, che sulla produzione e propagazione delle onde hanno la capillarità e i fenomeni dissipativi (attrito, viscosità, turbolenza, ecc.); in due o tre dimensioni, le onde provocate dal moto delle navi, i piccoli moti in canali molto profondi (cosiddette onde di Poisson-Cauchy), le oscillazioni in un bacino, le sesse lacuali, e, anche uscendo dal campo dei liquidi, tutta la teoria del suono e i fenomeni ondosi di grandi proporzioni oceanici e atmosferici, su cui ha influenza la rotazione della Terra.

15. Onde di discontinuità. - Nello stato attuale della meccanica e della fisica si può ancora dire che molti fenomeni naturali (e anzi, nel campo macroscopico, quasi tutti) trovano la loro rappresentazione matematica in sistemi S di equazioni alle derivate parziali, che involgono un certo numero di funzioni incognite $\phi$1, $\phi$2, .., $\phi$m (parametri fisici caratteristici del fenomeno), e nel caso di tre dimensioni, quattro variabili indipendenti (tre coordinate spaziali x1, x2, x3 e il tempo t); e questi sistemi S sono normali, cioè comprendono tante equazioni quante sono le funzioni incognite e risultano risolubili rispetto a certe derivate d'ordine massimo. Ogni particolare determinazione del fenomeno, vale a dire ogni soluzione del sistema S, resta univocamente individuata, quando, scelta ad arbitrio nello spazio quadridimensionale x1, x2, x3, t una varietà a tre dimensioni (o ipersuperficie) $\sigma$, si prefissano, pur esse ad arbitrio, le funzioni, cui debbono ridursi su $\sigma$ le incognite $\phi$i e le loro derivate fino a certi ordini (dipendenti dalla natura del sistema S). Ciò vale per una $\sigma$ generica; ma si possono avere particolari ipersuperficie $\gamma$, dette caratteristiche del sistema S, per le quali accade che gli elementi arbitrarî dianzi indicati (funzioni cui si riducono su $\gamma$ le $\phi$i e le loro derivate fino a quei certi ordini) non bastano più a individuare una soluzione di S, in quanto esistono infinite soluzioni, tutte soddisfacenti alle condizioni prescelte su $\gamma$. Se in tal caso si fissano, con opportune condizioni ulteriori, due di queste infinite soluzioni, p. es. $\phi$i, e $\phi$i*, le quali siano fra loro distinte, esse risultano, in generale, definite da entrambe le parti della ipersuperficie $\gamma$; ma se ci si limita a considerare la $\phi$i da una parte e la $\phi$i* dall'altra, si hanno due soluzioni del sistema S, le quali si raccordano traverso $\gamma$ soltanto parzialmente (cioè fino a quegli ordini, cui giunge l'arbitrarietà degli elementi iniziali su di una $\sigma$ generica), mentre al di là di tali ordini le due soluzioni presentano l'una rispetto all'altra, traverso $\gamma$, certi caratteri di discontinuità. Ora se dallo spazio quadridimensionale x1, x2, x3, t, si torna nello spazio ordinario x1, x2, x3, restituendo alla quarta variabile t il suo ufficio di parametro temporale, la ipersuperficie $\gamma$ dà luogo ad una superficie $\sigma$t, variabile nel tempo; e le due soluzioni $\phi$i, $\phi$i*,. definiscono per il fenomeno, di cui si tratta, due distinti regimi, validi rispettivamente dalle due parti opposte della $\sigma$t, la quale, istante per istante, segna fra quei due regimi una frontiera di parziale discontinuità. Al trascorrere del tempo questa frontiera $\sigma$t si sposta e, in generale, si deforma; e la conseguente propagazione della discontinuità nello spazio (e nel mezzo continuo, eventualmente legato al fenomeno) presenta la caratteristica fondamentale di un'onda, che, in contrapposto a quelle considerate fin qui, si qhiama appunto onda di discontinuità.

Per precisarne l'andamento si fissi sulla $\sigma$t relativa a un istante t, un punto P e la corrispondente normale n alla superficie. Quando si passa all'istante t + dt, la $\sigma$t assume una nuova configurazione $\sigma$t + dt, la quale sega la n in un punto Q prossimo a P, che definisce sulla n il senso di avanzamento dell'onda e, rispetto alla $\sigma$t (o, meglio, a una sua regione abbastanza piccola intorno a P), una faccia, che si chiama fronte d'onda.

Questa schematizzazione, che risale a Hugoniot e fu sistematicamente sviluppata da J. Hadamard, ha consentito di ridurre sostanzialmente a calcoli algebrici lo studio della propagazione delle onde acustiche, elastiche, elettromagnetiche e altre ancora (ma non quelle nei liquidi). Non è qui possibile insistere sugli sviluppi, per quanto relativamente elementari, che permettono di dedurre da un sistema normale S la definizione delle eventuali sue varietà caratteristiche, né sui criterî, che dànno il modo di seguire l'evoluzione di una superficie d'onda $\sigma$t a partire dalla sua configurazione iniziale. Basteranno alcune osservazioni.

Per quello che riguarda le varietà caratteristiche d'un sistema normale S, va rilevato che in ogni caso esse sono definite da un'unica equazione a derivate parziali E, che si deduce da S con un procedimento ben determinato e valido in ogni caso; ma può accadere che essa non ammetta alcuna soluzione reale. Per es., all'equazione del Laplace (v. armonico: Funzioni armoniche; funzione, n. 30; potenziale)

corrisponde, come equazione E di definizione delle rispettive varietà caratteristiche, la

che, nel campo reale, non è soddisfatta da alcuna funzione (salvo che da z = cost.); mentre per la cosiddetta equazione canonica dei piccoli moti (di un fluido incomprimibile)

l'equazione E è data dalla

la quale ammette soluzioni reali, dotate della massima generalità compatibile col suo ordine. Parlando in generale, si può aggiungere che l'equazione E delle varietà caratteristiche ha certamente soluzioni reali (e quindi esistono onde di discontinuità) per tutti i sistemi normali del cosiddetto tipo iperbolico, mentre l'opposto accade per quelli del cosiddetto tipo ellittico; e si presenta tutta una serie di casi intermedî per i sistemi, che non sono né completamente iperbolici, né completamente ellittici.

Quanto poi alle svariate classi di fenomeni, in cui il concetto di onda di discontinuità ha trovato feconde applicazioni, basterà fissare l'attenzione su due casi particolarmente interessanti:

1. Se si tratta d'un fenomeno relativo a un mezzo continuo omogeneo, si dimostra che, sotto opportune specificazioni rispecchianti il fatto fisico che anche altre eventuali cause concomitanti siano indipendenti dal posto, l'equazione E, che regge l'evoluzione della superficie oi dell'onda di discontinuità, è tale che questa superficie, se è inizialmente piana, si mantiene piana anche in ogni altro istante; cioè i mezzi omogenei ammettono in ogni caso onde di discontinuità piane.

2. Tornando al caso d'un fenomeno qualsiasi, si consideri l'eventualità, che la configurazione iniziale $\sigma$0 della $\sigma$t sia una piccolissima superficie chiusa, intorno ad un assegnato punto P, e traverso la $\sigma$t si raccordino parzialmente, nel senso chiarito poc'anzi, una soluzione interna diversa da zero e la soluzione identicamente nulla all'esterno. Si ha così la rappresentazione di quei fenomeni che s'iniziano con una perturbazione circoscritta alla immediata prossimità d'un determinato punto, detto epicentro; al variare del tempo il fronte d'onda va dilatandosi e la perturbazione rimane localizzata all'interno. Quando le caratteristiche fisiche del fenomeno sono costanti, come accade per le propagazioni sonore o luminose nei mezzi omogenei, le $\sigma$t relative ai successivi istanti sono tutte superficíe omotetiche rispetto all'epicentro; e, in particolare, se si tratta di propagazione luminosa in un mezzo biassico, le $\sigma$t epicentrali sono omotetiche alla superficie del Fresnel (n.11). Anche più interessante è il caso, in cui la regione perturbata (soluzione non nulla), anziché invadere tutto l'interno di $\sigma$0, e quindi poi di $\sigma$t, rimane circoscritta a uno straterello compreso fra la $\sigma$0 e un'altra superficie caratteristica interna ad essa e ovunque prossima; ciò corrisponde ai casi ordinarî d'un segnale sonoro o luminoso di breve durata.

In ogni caso un carattere fondamentale del fenomeno, che completa quantitativamente la nozione di fronte d'onda, è, in corrispondenza di ogni elemento d$\sigma$t della superficie $\sigma$t, relativa a un generico istante t, la velocità, con cui, rispetto al triedro di riferimento (osservatore fisso), l'elemento si sposta in direzione normale a sé stesso. È questa la cosiddetta velocità V di avanzanlento di quell'elemento di superficie d'onda. Se si tratta d'un fenomeno specificamente meccanico, cioè involgente, accanto al moto d'una superficie di discontinuità, quello d'un mezzo materiale (aria o altro gas o corpo solido), va presa in considerazione, oltre la velocità V, anche l'analoga velocità W, con cui l'elemento si sposta rispetto al mezzo. Essa si suole chiamare velocità di propagazione e, in forza del principio dei moti relativi (v. cinematica, n. 29), risulta legata alla V dalla relazione

dove vn denota la componente normale (alla superficie $\sigma$t, nel senso della propagazione) della velocità, da cui è animata la particella materiale, che, all'istante t, si trova sull'elemento d$\sigma$t considerato.

La distinzione fra la velocità di avanzamento e quella di propagazione, che è essenziale per i fenomeni tipicamente meccanici, non ha più ragione d'essere per quelli, che, come gli elettromagnetici, non implicano alcun mezzo materiale; e in tali casi si ha soltanto un'unica velocità, che a rigore è di avanzamento, ma, non essendovi pericolo di ambiguità, si designa, indifferentemente, anche col nome di velocità di propagazione.

Se si fissa una determinata categoria di fenomeni, con eventuali condizioni ai limiti, possono da un lato esistere o no onde progressive, e d'altro canto (secondo il tipo del corrispondente sistema normale) si possono avere o no onde di discontinuità. Quando sono possibili, a un tempo, onde di entrambe le specie, si è trovato, almeno nei casi studiati sinora, che la velocità V di avanzamento delle onde di discontinuità ammette quelle stesse espressioni per mezzo dei parametri fisici del fenomeno, che valgono per le corrispondenti onde progressive (n. 11).

Ma vi sono dei casi, in cui esiste uno solo dei due tipi di onde. Per es., i liquidi sono dotati di sole onde progressive (sistemi differenziali di tipo ellittico, o, quanto meno, più vicino a quello ellittico che all'iperbolico); mentre nei mezzi elastici omogenei di struttura generale (che dipendono da 21 costanti strutturali) non sono possibili onde piane progressive (E. Beltrami), ma esistono onde piane di discontinuità (G. Lampariello).

16. Onde e corpuscoli. - La teoria dianzi accennata delle onde di discontinuità apre l'adito a un'osservazione d'ordine generale sul dualismo fra la concezione corpuscolare e quella ondulatoria, che, nell'evoluzione della fisica, dopo una vicenda più volte rinnovata di sopravventi dell'una o dell'altra, tende oggi a comporsi, come fu detto fin da principio, in una veduta unitaria più elevata.

Nel campo dell'ottica si suole attribuire al Newton la concezione corpuscolare della luce e al Huygens quella ondulatoria, sebbene in realtà esse siano, almeno sotto qualche aspetto, molto più antiche e lo stesso Newton, pur dando la preferenza alla teoria dell'emissione, si valga talvolta d'immagini ondulatorie. Come già fu accennato, la grande contesa parve decisa, quando, dopo lo Young e il Fresnel, tutti i fenomeni fino allora conosciuti si poterono inquadrare in uno schema ondulatorio, fornito inizialmente da un modello elastico e successivamente dalle equi zioni elettromagnetiche del Maxwell. Ma poi non si riuscì a conciliare in modo semplice la teoria ondulatoria con i fatti osservati a proposito del fenomeno fotoelettrico, che risale a H. Hertz; sicché a rappresentare tali fatti fu necessario tornare alle vedute corpuscolari con la ipotesi quantistica di A. Einstein, secondo cui ogni fascio luminoso di data frequenza $\nu$ si considera costituito da uno sciame di fotoni (o quanti di luce), porziuncole E di energia, proporzionali ciascuna alla rispettiva frequenza, secondo la formula E = h$\nu$, dove h è la celebre costante di Planck. E più tardi questa stessa ipotesi permise di rendere conto dell'effetto Compton (1923), la cui natura complessa risulta soddisfacentemente spiegata, come rilevarono lo stesso Compton, P. Debye, E. Fermi, E. Persico, quando si associ all'ipotesi dell'Einstein, oltre il principio della conservazione dell'energia, quello della quantità di moto.

Una vicenda analoga, ma in senso inverso, si è presentata nella teoria degli elettroni. Mentre alla fine del secolo XIX, soprattutto in base al comportamento dei raggi catodici e alle celebri esperienze eseguite principalmente da J. J. Thomson, W. Kaufmann, H. A. Wilson, gli elettroni si ritenevano completamente caratterizzati come pure cariche elettriche, tutte eguali fra loro, questo punto di vista esclusivamente corpuscolare si mostrò insufficiente a rendere conto del fenomeno della diffrazione degli elettroni nei cristalli, scoperta nel 1927 da G. Davisson e L. H. Germer, e confermata da ulteriori esperienze, dovute a E. Rupp e G. P. Thomson; onde anche qui si presentò la necessità di ricorrere a una trattazione complementare di tipo oscillatorio.

Questo dualismo, per cui i più notevoli fatti della fisica moderna esigono l'intervento simultaneo di corpuscoli e di onde, fu riconosciuto e prospettato come legge generale di natura, anche prima della bella conferma desunta dalla diffrazione degli elettroni, da L. De Broglie, il quale, in un primo tempo cercò di dare forma più concreta alla sua concezione, associando ad ogni corpuscolo mobile un ben determinato gruppo o pacchetto d'onde. Ma egli stesso riconobbe le difficoltà provenienti da tale associazione.

Orbene, le considerazioni accennate al n. prec. sulle onde di discontinuità consentono di formulare uno schema matematico abbastanza largo per rispecchiare i due aspetti ondulatorio e corpuscolare d'un medesimo fenomeno, tutte le volte che questo trova un'adeguata rappresentazione in un sistema differenziale normale. Infatti all'equazione a derivate parziali del 1° ordine E, che definisce le superficie d'onda, è intrinsecamente associato (classico risultato di A. Cauchy) un determinato sistema differenziale ordinario C, di forma canonica, il quale definisce, nello spazio ordinario, $\infty$6 movimenti. Ognuno di questi è rappresentato nella varietà spazio-temporale x1, x2, x3, t, da una linea (oraria), che costituisce una bicaratteristica dell'equazione E e, attrai. erso questa, del sistema normale, che regge il fenomeno. In questo modo al fenomeno fisico considerato si associano simultaneamente un aspetto ondulatorio (onde di discontinuità) e un aspetto corpuscolare, quale appare schematizzato nei movimenti, di particelle reali o fittizie, lungo le bicaratteristiche con la legge oraria, che per ciascuna di esse è determinata dal sistema canonico C. Si ha, insomma, per qualsiasi teoria fisica (traducibile in un sistema differenziale normale), la possibilità di onde di discontinuità, risolubili matematicamente nel moto di particelle discrete.

Questa veduta è indubbiamente di carattere astratto e, fisicamente, agnostico; e, in particolare, non dà alcun criterio per un accoppiamento concreto dei due distinti ordini di apparenze. Molteplici tentativi in questa direzione, compiuti da scienziati eminenti, a cominciare, come si è detto, dallo stesso De Broglie, hanno portato alla conclusione, che non si può stabilire una corrispondenza biunivoca fra onde - o pacchetti d'onde - e corpuscoli, senza violare il cosiddetto principio d'indeterminazione di Heisenberg. Si può tuttavia allargare la questione e cercare di porre una legge di corrispondenza non più fra elementi individuali, ma fra treni d'onde da un lato e sciami di corpuscoli dall'altro. Si riesce così a stabilire una correlazione di tipo statistico, in base alla quale, con opportune precisazioni, si ritrova e si estende a casi più generali la celebre formula del De Broglie, che assegna (in assenza d'ogni altro fenomeno perturbatore) una ben determinata lunghezza d'onda $\lambda$ ad ogni tipo di raggi elettronici, concepiti sotto l'aspetto corpuscolare. Essa è data da

dove h è la costante del Plank e m e v denotano la massa materiale dei corpuscoli materiali costituenti il raggio e la loro velocità (media).

Bibl.: Sulla parte generale: H. Lamb, The dynamical Theory of Sound, 3ª ed., Londra 1929; Lord Raylegh, Theory of Sound, voll. 2, Londra 1894-1896; T. Levi-Civita, Questioni di meccanica classica e relativista, Bologna 1924; J. A. Fleming, Waves and ripples in water, air and aether, Londra 1912; V. Cornish, Ocean waves and kindred geophysical phenomena, Cambridge 1934. - Sulle onde progressive: A. E. H. Love, A treatise on the mathematical Theory of elasticity, 4ª ed., Cambridge 1927; G. A. Maggi, Teoria fenomenologica del campo elettromagnetico, Milano 1931; M. Abraham, Theorie der Elektrizität, I, 4ª ed., Lipsia 1912 (ora Abraham-Becker, I, 9ª ed., Lipsia 1932); M. Born, Optik, Berlino 1932 (specialmente capitoli 1-6); H. Lamb, Hydrodynamics, 6ª ed., Cambridge 1932; G. H. Darwin, The tides and kindred phenomena, 3ª ed., Londra 1911 (trad. it. di G. Magrini, La marea, Torino 1905); H. Poincaré, Leçons de mécanique céleste, III, Théorie des marées, Parigi 1910; T. Levi-Civita, Détermination rigoureuse des ondes permanentes d'ampleur finie, in Math. Annalen, XCIII (1925), pp. 264-314; J. Kampé de Fériet, Les rides, les vagues et la houle, in Revue des questions scientifiques, Lovanio 1932; J.-M. Burgers, Sur quelques recherches de Helmholtz et de Wien relatives à la forme des ondes se propageant à la surface de séparation de deux liquides, in Rend. Lincei, s. 6ª, V (1927); N. Kotchine, Détermination rigoureuse des ondes permanentes d'ampleur finie à la surface de séparation de deux liquides de profondeur finie, in Math. Annalen, XCVIII (1927); T. H. Havelock, Theory of ship waves and waveresistance, Londra 1926; E. Hogner, Contributions to the theory of ship waves, Stoccolma 1925; A. Tonolo, Nuova risoluzione del problema delle onde di Poisson-Cauchy, in Atti Istituto veneto, LXXIII (1914), pp. 545-571; A. Palatini, Sulla influenza del fondo nella propagazione delle onde dovute a perturbazioni locali, in Rend. Circ. matem. di Palermo, XXXXIX (1915), pp. 161-184; F. Vercelli, La teoria delle sesse, ecc. in Memorie del R. Istit. lomb. XXI (1909); V. Bjerknes, J. Bierknes, H. Solberg, Hydrodynamique physique, voll. 3, Parigi 1934. - Sulle onde di discontinuità: T. Levi-Civita, Caratteristica dei sistemi differenziali e propagaz. ondosa (lezioni raccolte da G. Lampariello), Bologna 1931; id., Alcuni aspetti della nuova meccanica, in Nuovo cimento, XI (1934).

Fenomeni fisici di carattere ondoso.

17. Onde elastiche. - Propagazione. - Se una parte materiale di un corpo elastico (solido, liquido, gassoso) piccola rispetto alle dimensioni di questo, per una qualunque causa viene ad allontanarsi dalla sua posizione naturale di quiete, l'equilibrio che sussisteva internamente in questo corpo (mezzo) viene ad essere alterato e questa alterazione (perturbazione) non si limita in genere all'elemento materiale di cui sopra, ma, per forze di natura elastica interne al mezzo, si comunica (si propaga) a quelli immediatamente vicini che vengono così alla loro volta perturbati dal loro stato di equilibrio. Alla loro volta questi elementi trasmettono o determinano la trasmissione della perturbazione a quelli a loro vicini, e così di elemento in elemento, e in tutte le direzioni accessibili, la perturbazione si propaga nel corpo con una velocità V più o meno grande, che dipende dalla rapidità con cui si manifestano le azioni elastiche nell'interno del mezzo.

Di particolare interesse è il caso che in un punto (sorgente) di un mezzo venga eccitata e mantenuta per un certo tempo un'oscillazione periodica. Allora, se il mezzo è tale da propagare tali oscillazioni come perturbazione, se cioè i varî elementi del mezzo sono capaci d'agire l'uno sull'altro, questa oscillazione della sorgente si trasmette progressivamente a tutti gli elementi accessibili del mezzo stesso, e ciascuno di questi, prima o poi (a seconda della sua posizione), entra in oscillazione con la frequenza della sorgente. Si dice allora che dalla sorgente è stata emessa un'onda elastica.

La velocità di propagazione di un'onda, per un certo tipo di onde in un determinato mezzo, è una grandezza caratteristica del corpo in esame, ed è definita, al solito, come il rapporto fra il percorso s della perturbazione ondosa e il tempo t impiegato a percorrerlo per cui s = Vt. Essa dipende essenzialmente dalla prontezza con cui un elemento perturbato può agire su quelli vicini.

Nel caso delle onde elastiche di tale processo ci si può fare un'immagine concreta, per quanto grossolana, ragionando su un modello che ci sarà utile anche per il seguito. Riferendoci per semplicità a una sola dimensione, consideriamo come mezzo una serie di particelle materiali (e quindi aventi una certa massa), le quali possano vibrare solo trasversalmente alla retta AB (fig. 5). Le forze di interazione fra questi punti, quelle tali forze che determinano la propagazione, siano invece delle piccole molle tese (di massa trascurabile) che collegano fra loro le varie particelle. Se, per es., s'inizia la perturbazione imprimendo alla prima particella materiale P una certa velocità, questa sollecita per mezzo della molla quella immediatamente vicina a fare altrettanto, e questa la segue infatti nel suo movimento, ma con un certo ritardo determinato dal fatto che la forza esercitata dalla molla ha impresso alla particella materiale non una velocità finita, ma un'accelerazione che permette alla medesima d'assumere gradualmente una determinata velocità. Questo ritardo, questo sfasamento che si ha fra il primo e il secondo elemento, si ha naturalmente fra il secondo e il terzo, fra questo e quello che lo segue e così via, ed è evidentemente una conseguenza dell'inerzia delle particelle vibranti. Nella fig. 5 sono rappresentate le prime fasi di questo processo, fino al momento in cui la sorgente effettuato il suo ritorno alla posizione d'equilibrio l'ha oltrepassata iniziando per inerzia l'oscillazione in senso opposto, mentre altre particelle successive seguitano ancora la loro corsa nel verso primitivo.

Ora è proprio questo graduale ritardo ad entrare in vibrazione, che si stabilisce tra un elemento e quelli vicini, che fa sì che la velocità di propagazione sia finita e che le perturbazioni non si propaghino in un mezzo materiale con velocità infinitamente grande come accadrebbe se l'inerzia degli elementi fosse nulla (e quindi la densità media degli elementi fosse ugualmente nulla).

18. Onde elastiche longitudinali e trasversali. - Il movimento delle particelle perturbate in un'onda può essere o un movimento di va e vieni sulla linea di propagazione o un movimento normale a questa linea. Un'onda del primo tipo si dice longitudinale, un'onda del secondo trasversale. È possibile generare onde longitudinali elastiche in tutti i corpi: le trasversali solo nei corpi non fluidi. In un mezzo fluido sono possibili infatti solo onde longitudinali, che non sono altro che un susseguirsi di stati di compressione e rarefazione. In un mezzo solido sono invece sempre presenti anche le onde trasversali che sono determinate da fenomeni di elasticità di flessione e di torsione.

La velocità di propagazione delle onde longitudinali è nei corpi solidi e liquidi data da

dove E è il modulo di elasticità e $\rho$ la densità media del mezzo solido o liquido. Come si vede da questa formula (e come avevamo già visto intuitivamente col modello di cui sopra) in un mezzo elastico di densità molto piccola la velocità può essere grandissima. Ugualmente può essere molto grande questa velocità in un mezzo elastico poco compressibile, ad esempio in un liquido. Per i liquidi l'equazione (12) si scrive spesso anche sostituendo a E il suo inverso K =1/E che si chiama modulo di compressibilità. L'equazione (12) prende allora la forma

Per l'acqua, per esempio, K = 4,8 • 10-11 (C.G.S.), $\rho$ = 1, e quindi

Per le onde trasversali vale una formula del tutto analoga

dove $\mu$ indica il modulo di rigidità. Poiché $\mu$ è sempre minore di E, nei mezzi dove si propagano onde longitudinali e trasversali, le prime hanno una velocità superiore alle seconde.

Nei terremoti che sono delle onde elastiche che si propagano alla superficie e nell'interno della Terra, possono aversi sia delle onde trasversali come delle onde longitudinali. Queste ultime arrivano, per la loro diversa velocità, nei varî punti della Terra in tempi diversi. Tanto più grande è la distanza della sorgente del terremoto (epicentro) tanto maggiore è questa differenza di tempo. Si può da questa differenza calcolare allora, con una certa approssimazione, questa distanza.

19. Onde acustiche. - Fra le onde elastiche longitudinali quelle che hanno maggiore interesse dal punto di vista fisico sono le onde acustiche o sonore. Per suono s'intende qualunque processo fisico atto a eccitare il nostro orecchio. La sorgente d'un suono è in generale un corpo vibrante che genera nel mezzo materiale circostante (gas, liquido) delle onde longitudinali propagantisi più o meno rapidamente a seconda della natura del mezzo stesso. Noi percepiamo il suono quando queste onde raggiungono il nostro timpano, e lo pongono in vibrazione con la stessa frequenza della sorgente. Essendo le onde sonore delle onde di natura elastica, la loro velocità di propagazione è caratterizzata dalle qualità elastiche del mezzo in cui si propagano (v. suono).

Per il caso dei gas, ammettendo (come realmente si verifica in pratica) che le compressioni e rarefazioni si susseguano in un determinato luogo al passaggio di un'onda sonora, in modo adiabatico (cioè senza dispersione sensibile d'energia), dalla formula (12) si giunge facilmente, in questo caso particolare, alla seguente

dove p è la pressione media del gas, $\rho$ la sua densità, e $\gamma$ il rapporto Cp//Cv fra il calore specifico a pressione costante e quello a volume costante del gas. Per un gas monoatomico come l'elio, l'argon, ecc., è $\gamma$ = 1,67; per un gas biatomico, azoto, ossigeno, ecc., è $\gamma$ = 1,40; per i gas poliatomici è circa 1,33; per l'aria in particolare si può prendere $\gamma$ = 1,40; e allora, in accordo con i dati sperimentali, si trova per la pressione normale a 0° 331,5 m./sec., a 20° 340 m./sec., ecc., all'ingrosso in media 1/3 di km./sec.

La velocità del suono si può misurare nei modi più svariati, da quelli d'uso comune, che utilizzano l'eco o la segnalazione luminosa dell'istante in cui s'inizia la propagazione d'una detonazione, a quelli molto più precisi in cui si ricorre a fenomeni d'interferenza (onde stazionarie) per determinare la lunghezza d'onda $\lambda$ d'una certa nota di frequenza $\nu$ per ricavare poi V dalla formula V = $\lambda$$\nu$.

20. Sorgenti acustiche. - Poiché una sorgente acustica può essere qualunque corpo vibrante, si hanno a seconda delle vibrazioni possibili alle sorgenti stesse: a) sorgenti sonore a una sola dimensione, o che si possono considerare con buona approssimazione come tali (corde, aste, diapason, tubi sonori); b) sorgenti a due dimensioni (membrane vibranti, piatti, campane); c) sorgenti a tre dimensioni (casse sonore). Spesso, per intensificare un suono, si accoppiano per risonanza sorgenti diverse.

A queste vanno aggiunte le sorgenti in cui il susseguirsi di compressioni e rarefazioni, che costituisce l'onda sonora, non viene determinato dalle vibrazioni d'un corpo elastico, ma direttamente interrompendo a intervalli di tempo regolari un forte getto d'aria.

Un disco, girevole intorno al suo asse, in cui siano stati praticati, a distanza costante l'uno dall'altro, dei fori situati lungo una circonferenza coassiale al disco, posto in rapida rotazione, mentre un forte getto d'aria lo investe via via lungo i fori suddetti, dà luogo a un suono il quale è tanto più acuto quanto più rapida è la rotazione. È un tale apparecchio che si chiama, come è noto, sirena. Esso, per la relativa facilità con cui si può regolare (regolando la velocità di rotazione) l'altezza del suono, è molto usato nelle ricerche di acustica. Accanto ad esso, per la regolarità delle vibrazioni di cui è capace, è da mettere (in particolare se eccitato elettromagneticamente) il diapason.

L'orecchio ha una sensibilità limitata a un determinato intervallo di frequenze. La più bassa nota percepibile è di circa 16 vibrazioni al secondo; la più elevata frequenza udibile è di circa 40.000. Le frequenze superiori non udibili dall'orecchio umano costituiscono i cosiddetti ultrasuoni (v.). La voce umana raggiunge, nei bimbi, circa le 20.000 vibrazioni al secondo, le note musicali sono in genere comprese fra 16 e 4000 vibrazioni al secondo. Le lunghezze d'onda relative vanno da qualche diecina di metri al centimetro.

L'eccitazione delle sorgenti acustiche viene fatta essenzialmente o per via meccanica o per via elettrica (magnetica o elettromagnetica). Esistono anche metodi termici d'eccitazione. In ogni caso la sorgente o può essere mantenuta in vibrazione (con cessione di energia durante tutto il tempo dell'eccitazione) per un determinato tempo o può essere eccitata una volta e poi abbandonata a sé (per es., con un colpo dato da un martelletto). Nel primo caso si dice di avere un suono permanente, nel secondo un suono smorzato.

I mezzi d'eccitazione meccanica sono essenzialmente basati sull'urto o sull'attrito. Molto più usate, a scopo di ricerca scientifica, sono invece le sorgenti eccitate elettricamente. Per via elettrica l'eccitazione può essere fatta nei modi più svariati, da quello in cui la forza eccitante è determinata da un'elettrocalamita agente, per es., su una membrana magnetizzabile (telefono), a quello puramente elettrico in cui si utilizzano (per altissime frequenze) le proprietà piezoelettriche del quarzo (v. piezoelettricità).

Come nelle sorgenti acustiche si trasforma dell'energia meccanica, elettrica, elettromagnetica, in energia di natura meccano-elastica propagantesi per onde, viceversa nei ricevitori delle onde stesse s'inverte il processo e nella maggior parte dei casi a ogni sorgente sonora corrisponde un ricevitore basato sullo stesso principio. Esempio tipico il comune telefono elettromagnetico.

21. Riflessione, diffrazione, interferenze di onde sonore. - Le onde sonore si riflettono e si rifrangono come ogni altro tipo di onde secondo le leggi che si deducono dal principio di Huygens (v.).

Nelle applicazioni pratiche la cosidetta "acustica dei locali" dipende essenzialmente dal potere riflettente delle pareti del locale stesso. Pareti imbottite, tende pesanti riflettono malamente qualunque suono, smorzandolo fortemente. Nei grandi locali con pareti piane molto vaste la riflessione è invece molto intensa e per la notevole differenza di tempo fra suono diretto e eco può provocare dei disturbi musicali molto sgradevoli.

Le onde sonore sono naturalmente anche capaci d'interferire. Si può mettere in evidenza tale fenomeno utilizzando due onde emesse da una stessa sorgente e facendole poi sovrapporre con opportuni dispositivi in un determinato punto. È un'esperienza classica, in questo senso, quella di G. H. Quincke (fig. 6). Il ramo ACB è incastrato nel ramo fisso ADB e può scorrere in questo in modo da poter variare, entro un certo intervallo, la sua lunghezza. In A viene posta una sorgente sonora, in B un ricevitore (per es. l'orecchio). Se il tubo ACB è così estratto da quello fisso ADB che la differenza nella lunghezza dei due sia uguale a una mezza lunghezza d'onda, allora le due onde propagantisi nei due rami giungono in B in opposizione di fase e l'intensità del suono percepibile è minima. Si possono in questo modo eseguire delle misure di lunghezza d'onda d'un suono nell'aria.

In tutti i fenomeni d'interferenza si realizzano naturalmente delle onde stazionarie, ma esse si pongono, nella loro caratteristica distribuzione d'intensità, particolarmente in evidenza facendo riflettere le onde sonore su una parete piana. Nello spazio posto fra la parete e la sorgente, dove si sovrappongono l'onda incidente e quella riflessa (che ritorna verso la sorgente stessa) si stabiliscono in effetti le onde stazionarie dovute a questa sovrapposizione. Si possono allora, spostandosi dalla parete verso la sorgente, percepire nettamente anche ad orecchio (meglio con delle fiamme vibranti) i minimi e i massimi d'intensità che caratterizzano appunto le onde stazionarie. Dalla distanza fra un massimo e un minimo, che come è noto è di un quarto di lunghezza d'onda, si può dedurre anche in questo caso la lunghezza d'onda d'una determinata nota. Questo metodo è stato applicato da A. Kundt per determinare la velocità delle onde sonore nei gas esaminando la distribuzione assunta da grani di polveri sottili nell'interno di tubi (chiusi da una parete riflettente) in cui venivano stabilite delle onde stazionarie. I grani, in principio distribuiti uniformente, si raccolgono allo stabilirsi delle onde stazionarie nei punti del tubo in cui le vibrazioni sonore sono meno intense e rivelano la posizione dei minimi d'intensità.

22. Battimenti. - I battimenti acustici si verificano quando due note di frequenza poco diversa si sovrappongono. Si osservano facilmente con due diapason uguali di cui uno abbia un estremo appesantito, per es. da un po' di cera. I battimenti si manifestano come un più o meno rapido alternarsi nel tempo di massimi e minimi d'intensità. La rapidità dei battimenti dipende dalla differenza tra le due frequenze; più grande è questa differenza, più rapido è l'alternarsi dei massimi e minimi, ma al tempo stesso tanto meno netto è il fenomeno.

23. onde soniche. - Alle onde elastiche da un lato e a quelle acustiche dall'altro vanno ravvicinate le cosiddette onde soniche. Si designano con questo nome quelle rapide alternanze periodiche di pressione prodotte artificialmente nell'acqua contenuta in una tubazione a fine di "travasare" energia cinetica da un estremo all'altro della tubazione stessa (qualche decina di cavalli).

Il liquido è fortemente compresso (un migliaio di atmosfere) ed è sede di movimenti elastici, in modo che le rapide variazioni di pressione, prodotte dall'alterno movimento d'uno stantuffo, si propagano lungo la colonna liquida con celerità analoga a quella del suono (circa 1400 m. al secondo), per giungere all'altro estremo del tubo, dove un secondo stantuffo è costretto a muoversi sincronicamente al primo, eseguendo un lavoro utile. Per opportune frequenze si hanno fenomeni analoghi a quelli della risonanza acustica ed elettrica, di guisa che allorché l'insieme è eccitato con frequenza opportuna o si "accorda" su quella propria, le pressioni si esaltano fino a provocare la rottura dell'involucro. Sembra accertato che i metalli, le rocce e in genere le sostanze solide modifichino le loro proprietà allorché sono sottoposte ad alte pressioni rapidamente alternate; industrialmente quindi le onde soniche trovarono utile applicazione nei martelli a rapida successione di percussione, sia per officina sia per le perforatrici da galleria e da cave. Le onde soniche furono inventate durante la guerra mondiale dall'ingegnere romeno G. Constantinescu, che le applicò per ottenere nei monoplani la sincronizzazione dei colpi delle mitragliatrici con i giri dell'elica di trazione.

24. Onde elettromagnetiche. - Qualitativamente e formalmente considerate, le due leggi fondamentali di Maxwell si possono enunciare come segue:

a) Attorno alle linee di forza d'ogni campo elettrico variabile nel tempo si manifestano sempre delle linee di forza magnetica che si chiudono intorno a quelle di forza elettrica e gli sono ortogonali;

b) le linee di forza d'ogni campo magnetico variabile nel tempo sono sempre circondate da linee di forza elettrica che si chiudono intorno a quelle magnetiche e sono a loro ortogonali.

La produzione di questi campi elettromagnetici da parte di altri campi elettromagnetici non è istantanea. I campi elettromagnetici manifestano un'inerzia a stabilirsi analogamente a quanto accade nel caso delle perturbazioni elastiche. Quest'inerzia elettromagnetica è quella che fa sì che un campo elettromagnetico variabile nel tempo si propaghi con una velocità finita.

È facile rendersi conto, sia pure in modo molto grossolano, di come si possano creare delle onde elettromagnetiche, cioè dei campi elettromagnetici rapidamente variabili nel tempo e quindi propagantisi nello spazio.

Consideriamo la sorgente più semplice di onde elettromagnetiche; un'asta metallica in cui siano provocate con uno dei dispositivi di cui faremo cenno in seguito, delle oscillazioni elettriche, ossia una corrente alternata di frequenza elevata. All'inizio delle oscillazioni una parte degli elettroni che inizialmente era in equilibrio nel filo completamente scarico in ogni sua parte, viene sospinta verso un estremo di questo filo che così si carica negativamente, mentre l'altro estremo viene ad assumere una carica positiva. Le due cariche così ottenute cercano allora di annullarsi e gli elettroni in eccesso all'estremo negativo sono accelerati nella direzione dell'estremo positivo. Però non si fermano quando l'equilibrio è raggiunto perché sospinti dalla propria inerzia gli elettroni si accumulano ora dove prima erano in difetto, in modo che l'estremo prima carico positivamente si carica ora negativamente. Il giuoco s'inizia allora in senso inverso e prosegue, e in questo oscillatore o, come si suole dire, in questo dipolo si hanno delle oscillazioni elettriche che sono smorzate (anche se il dipolo non ha resistenza apprezzabile e quindi non dissipa energia per effetto Joule) se il dipolo è abbandonato a sé e sono invece persistenti se, con continua cessione di energia, le oscillazioni sono mantenute.

Consideriamo ora l'andamento delle linee di forza elettrica intorno a questo dipolo oscillante. Le linee di forza elettrica sono sempre dirette dalle cariche positive verso le cariche negative e quindi possiamo fino da ora asserire che nel nostro caso esse cambieranno alternativamente direzione fra un estremo e l'altro del dipolo. Contemporaneamente e gradualmente anche il campo ad esse connesso cambierà d'intensità (e le linee di forza, secondo le convenzioni ad esse relative, di densità) da un massimo quando la dissimmetria nella distribuzione delle cariche è la più forte possibile a un minimo negli istanti in cui nel dipolo si ristabilisce l'equilibrio. Ebbene (questi campi) queste linee di forza elettrica rapidamente variabili nel tempo non si localizzeranno permanentemente intorno al dipolo smorzandosi ogni volta che nel dipolo si ristabilisce l'equilibrio e riformandosi poi gradualmente al ristabilirsi delle cariche opposte agli estremi di questo perché, per le leggi di Maxwell, ad esse si concateneranno (dei campi) delle linee di forza magnetica (da distinguersi da quelle dovute alla corrente variabile che pulsa nel dipolo) anch'esse variabili d'intensità in relazione alla rapidità di variazione delle linee di forza elettrica.

Ora queste linee di forza magnetica, variabili nel tempo, alla loro volta determinano delle linee di forza elettrica variabili che tornano a concatenarsi con altre linee di forza magnetica e così via man mano procedendo in questa effettiva propagazione d'un campo elettromagnetico variabile; questa propagazione avviene con velocità finita, perché, come abbiamo già detto più sopra, i campi elettromagnetici manifestano una certa inerzia a stabilirsi.

Dopo un certo tempo dall'inizio del fenomeno, tutto lo spazio circondante il dipolo è sede di campi elettrici e magnetici le cui linee di forza si concatenano reciprocamente. In ogni punto il campo magnetico e il campo elettrico sono diretti perpendicolarmente fra loro. Naturalmente data la forma del dipolo l'intensità di questi campi anche in un determinato istante non è costante lungo una generica linea di forza. La fig. 7 dà una rappresentazione schematica delle linee di forza elettrica intorno a un dipolo in un determinato istante. Nella fig. 8 invece sono rappresentate cinque fasi dell'irradiazione di un dipolo: a) le oscillazioni elettriche del dipolo non hanno ancora avuto inizio, il dipolo è completamente in equilibrio; b) dopo un quarto di periodo dall'inizio delle oscillazioni si hanno agli estremi le cariche massime; c) il dipolo durante il seguente quarto di periodo. L'equilibrio si sta ristabilendo nella parte centrale del dipolo, è anzi già raggiunto, le linee di forza hanno iniziato la propagazione e si chiudono in parte nello spazio; d) alla fine del secondo quarto di periodo quando l'equilibrio è gia raggiunto; e) dopo tre quarti di periodo quando la distribuzione delle cariche si è invertita.

Poiché un campo elettromagnetico è sede d'energia, il dipolo irradiante onde elettromagnetiche, irradia anche energia. Anzi è proprio per chiarire la nozione di questo irraggiamento d'energia, che è stato introdotto il concetto di onda elettromagnetica. Come è noto, nel vuoto (e praticamente nell'aria) le onde elettromagnetiche e l'energia ad esse collegata si propagano con una velocità di circa 300.000 km./sec. L'energia di questo campo elettromagnetico irradiato è naturalmente sottratta al dipolo, il quale, appunto per questa ragione, anche se non dissipa energia per effetto Joule, se abbandonato a sé è sede di un'oscillazione che va gradatamente smorzandosi. Questo tipo di smorzamento si chiama smorzamento per irradiazione.

Nel caso del dipolo questo smorzamento è particolarmente intenso e in generale lo stesso accade per tutti i cosiddetti circuiti oscillanti aperti. Lo smorzamento è . invece piccolo nei circuiti oscillanti chiusi, come p. es. il circuito costituito da un condensatore chiuso su sé stesso attraverso un'autoinduzione. Evidentemente per realizzare delle sorgenti di onde elettromagnetiche si utilizzano circuiti aperti come il dipolo di cui sopra. Per alimentare invece le sorgenti stesse si accoppiano i circuiti aperti con circuiti chiusi. Il circuito aperto prende allora il nome di antenna trasmittente.

Quando un sistema elettromagnetico sta irradiando, in ogni punto dello spazio accessibile alle onde elettromagnetiche da questo emesse sono presenti un campo elettrico e uno magnetico, alternati.

Un conduttore posto in questo punto dello spazio diviene allora sede di correnti alternate determinate in esso da questi campi. Queste sono delle correnti d'induzione per quanto riguarda il campo magnetico, delle correnti di conduzione (chiuse all'esterno dalle cosiddette correnti di spostamento) per quanto riguarda il campo elettrico. Un tale conduttore diviene allora un ricevitore di onde elettromagnetiche (antenna ricevente).

Naturalmente tanto le antenne trasmittenti, quanto quelle riceventi si costruiscono in pratica in modo che le loro qualità di trasmissione e di ricezione siano massime. Per es., si esalta la loro autoinduzione con apposite bobine inserite nel circuito dell'antenna, ecc. Anche le antenne riceventi si accoppiano poi con dei circuiti chiusi atti a entrare in risonanza con le antenne stesse in modo da amplificarne le correnti.

25. Sistemi trasmittenti e sistemi riceventi. - Dalle prime esperienze di Hertz fino ai moderni impianti di trasmissione e ricezione radiofonica su onde della più varia lunghezza, il cammino per quanto rapidissimamente percorso è stato assai lungo.

Hertz, nelle sue celebri esperienze che iniziarono quella serie di conquiste scientifiche che hanno condotto ad unificare in un unico spettro (fig. 9) tutte le radiazioni elettromagnetiche oggi conosciute, dalle onde hertziane di qualche km. di lunghezza d'onda ai raggi $\gamma$ la cui lunghezza d'onda può essere di frazioni di unità X (10-11 cm.), usò degli oscillatori e dei ricevitori in cui le oscillazioni elettriche erano date e rivelate per mezzo di scintille oscillanti, ottenute facendo scaricare delle piccole capacità in circuiti appositamente muniti di spinterometri. Le prime stazioni trasmittenti di onde radiotelegrafiche utilizzarono ugualmente dei circuiti oscillanti costituiti da un sistema induttanza e capacità connesso, come si è sopra accennato, con un'antenna trasmittente. Lo smorzamento di tali oscillazioni essendo fortissimo, anche con un susseguirsi rapidissimo di scariche, le onde dovute ad una di esse erano completamente distinte da quelle dovute alla successiva e si rendeva così possibile la trasmissione per alfabeto Morse. Oggi questi sistemi sono completamente abbandonati e i trasmettitori pur rimanendo essenzialmente costituiti da un circuito chiuso accoppiato con un'antenna, sono tutti a onde non smorzate con sistemi di valvole termoioniche accoppiate opportunamente con induttanze e capacità in modo da far sì che, automaticamente, una volta eccitate, le oscillazioni elettriche si mantengano persistentemente nel circuito chiuso e quindi vengano continuamente emesse dall'antenna.

Le lunghezze d'onda utilizzate in radiotelegrafia e radiotelefonia variano fra 0,3 e 15 km. ma ora queste lunghezze d'onda si stanno gradualmente abbandonando, in vista dei grandi vantaggi offerti, specialmente per trasmissioni a grandi distanze, dalle onde corte (10-50 m.).

Le antenne si allontanano poi in pratica, molto spesso, dalla loro tradizionale forma di mezzo dipolo essendo munite in testa di telai o di reti che permettono di guadagnare in spazio ed in intensità.

Ancora più che nei sistemi trasmittenti le valvole termoioniche hanno trovato la più vasta applicazione nei sistemi riceventi; anzi le valvole a tre elettrodi sono nate appunto nel felice tentativo di amplificare e rivelare le piccolissime correnti con cui si aveva a che fare nei sistemi riceventi radiotelegrafici. Si può oggi senza eccessive difficoltà e con antenne normali rendere sensibili delle onde elettromagnetiche che raggiungono l'antenna ricevente con un'intensità di 310-7 volt per cm. e, per mezzo delle onde corte, si può oggi far sì che le onde d'una sorgente ritornino al luogo d'origine e qui possano essere rivelate dopo avere percorso tutta la superficie della terra in circa o,13 sec.

26. Lo spettro totale delle onde elettromagnetiche. - Come abbiamo già detto, le onde hertziane, da quelle della T.S.F. a quelle di pochi centimetri degli oscillatori di Hertz, sono solo una parte delle onde elettromagnetiche conosciute. Altrettanto si dica delle onde luminose ossia delle onde elettromagnetiche a cui il nostro occhio è sensibile. Le onde luminose sono anzi una piccola parte dello spettro delle onde elettromagnetiche e la cosa appare in modo assai evidente se si osserva la fig. 9 in cui sono riportati i logaritmi delle lunghezze d'onda in unità Angström (10-8 cm.) insieme con i limiti che definiscono e differenziano i varî dominî pertinenti ai varî tipi di onde elettromagnetiche.

Pertanto (vedi, ad es., le onde hertziane) è stato possibile, con appropriati dispositivi, mettere in evidenza che lo spettro delle onde elettromagnetiche si estendeva anche al di là del limite visibile e anzi che si estendeva quasi illimitatamente, sia verso le grandi lunghezze d'onda come verso quelle brevissime. E non esiste nemmeno nessuna ragione teorica la quale neghi la possibilità d'esistere ad una qualsiasi lunghezza d'onda compresa fra $\lambda$ = $\infty$ ($\nu$ = 0) e $\lambda$ = 0 ($\nu$ = $\infty$).

Le radiazioni di lunghezza d'onda più grande delle radiazioni rosse si dicono appunto infrarosse e lo spettro infrarosso è limitato dalla parte delle grandi lunghezze d'onda dalle onde hertziane o onde T.S.F. di cui abbiamo già parlato.

Dalla parte del violetto, verso le piccole lunghezze d'onda, s'incontrano prima lo spettro dell'ultravioletto, poi i raggi Röntgen (X), e infine i raggi $\gamma$. Al di là di questi e precisamente al di là della radiazione $\gamma$ più dura che si conosca con certezza (che è quella del Th C' di 4,7 unità X) lo spettro si estende ancora molto con grande probabilità, anche con radiazioni che sono già oggetto d'indagini sperimentali (radiazioni ultragamma), ma per ora non si hanno per questa asserzione prove sufficienti. Nella seguente tabella sono dati approssimativamente le estensioni dei singoli dominî spettrali in unità A.

Le leggi che valgono per le onde luminose e che caratterizzano i cosiddetti fenomeni ottici valgono in generale per tutto lo spettro delle onde elettromagnetiche. Spesso, però, l'aspetto esteriore dei fenomeni è diverso da quello che i fenomeni stessi manifestano nello spettro visibile, poiché cambiano in modo notevolissimo le proprietà dei corpi rispetto alle diverse radiazioni.

Per es. la trasparenza del vetro per le radiazioni luminose è subito dopo seguita da una più o meno perfetta opacità per le radiazioni ultraviolette. Per questo, non appena ci si allontana dallo spettro visibile, sono necessarî strumenti dotati di lenti, prismi, ecc., fatti di sostanze opportune (quarzo per l'ultravioletto, salgemma per il primo infrarosso, ecc.).

Il principio delle misure di lunghezza d'onda è tuttavia sempre lo stesso ed è sempre basato direttamente o indirettamente sulla possibilità d'interferire delle onde stesse.

27. Metodi di studio e di rivelazione delle onde nei varî domini dello spettro. - Se due corpi a temperatura diversa stanno l'uno di fronte all'altro ma non in contatto materiale (per es., in un recipiente dove sia praticato il vuoto) così che sia impossibile qualunque conduzione diretta di calore, accade ugualmente che, dopo un certo tempo più o meno lungo, le due temperature si uguagliano. Ciò accade per irraggiamento reciproco dei due corpi e precisamente per irraggiamento di natura elettromagnetica (calore raggiante).

Questo livellamento di temperatura per irraggiamento consiste nel fatto che il corpo caldo invia al corpo freddo delle radiazioni che ivi giunte si convertono in calore, in misura maggiore di quanto ne producano quelle che dal campo freddo vanno a quello caldo.

Questi scambî d'energia fra radiazione e materia sono utilizzati per misurare l'energia (l'intensità) della maggior parte delle radiazioni elettromagnetiche. In pratica, per rivelare il calore che proviene dalla trasformazione dell'energ a di una radiazione in energia termica s'impiegano degli elementi termoelettrici montati su galvanometri. Tali apparecchi possono raggiungere grandissime sensibilità e possono servire a rivelare e a studiare radiazioni di debolissima intensità (microradiometri).

Molto spesso, per quanto in modo non molto felice, si considerano come calore raggiante più in particolare le radiazioni infrarosse, ma è da ricordare che un corpo caldo (teoricamente, che non sia allo zero assoluto) emette sempre non solamente delle radiazioni infrarosse, ma anche visibili e ultraviolette, e che come calore raggiante è da intendere ogni radiazione che sia determinata dalla temperatura di un corpo.

Comunque, il modo più comodo d'eccitare delle radiazioni infrarosse è precisamente quello termico. Per studiarle, come abbiamo già accennato, si ricorre a strumenti le cui parti ottiche sono, secondo il particolare dominio di spettro in esame, di quarzo (fino a 4 $\mu$ di lunghezza d'onda), di spato (fino a 8,5 $\mu$), di salgemma (fino a 14 $\mu$) e di silvina (fino a circa 20 $\mu$).

Per la limitazione d'una piccola regione dello spettro infrarosso, limitazione necessaria quando si voglia studiare in particolare il comportamento fisico di radiazioni infrarosse d'una determinata lunghezza d'onda, si utilizza generalmente il metodo dei cosiddetti raggi restanti o di Rubens.

Esistono molte sostanze (come la calcite, il salgemma, il bromuro di potassio, il bromuro di tallio, ecc.) che hanno per certe particolari lunghezze d'onda infrarosse un forte potere riflettente, nettamente superiore al potere riflettente che manifestano per radiazioni di lunghezza d'onda poco diversa. Facendo riflettere più volte su tali superficie una radiazione infrarossa comprendente un notevole dominio di lunghezze d'onda si ottiene allora, dopo la riflessione multipla, un raggio emergente limitato sensibilmente solo a un brevissimo intervallo di frequenza. Tale raggio si chiama appunto raggio restante (Reststrahl).

Come per i raggi infrarossi così anche per quelli ultravioletti occorrono in genere strumenti costruiti con apposite sostanze che sono ancora quarzo, salgemma, ecc. La maggior parte delle sostanze però è praticamente opaca per le radiazioni ultraviolette di lunghezza d'onda molto breve. Anzi risiede proprio in questa grande assorbibilità delle radiazioni ultraviolette di grande frequenza, la causa maggiore delle difficoltà che s'incontrano nello studio di esse. Si deve a Schumann e a Millikan se la lacuna esistente fra radiazioni ultraviolette e raggi X è stata completamente colmata. Anche l'aria assorbe fortemente le radiazioni ultraviolette e per questa ragione in tutti gli spettrografi e apparecchi destinati allo studio di queste radiazioni si pratica il vuoto. Per la rivelazione delle radiazioni ultraviolette si utilizza talvolta il loro potere di rendere fosforescenti certe particolari sostanze o il loro potere di ionizzare i gas e spesso di impressionare le lastre fotografiche per il forte potere attinico delle radiazioni ultraviolette stesse. Tale potere attinico non può in generale essere sfruttato per le fotografie normali e specialmente per le fotografie a distanza, a causa dell'assorbimento atmosferico di cui si è detto. Anzi per le fotografie a distanza si stanno ora adoperando con grande successo apposite lastre sensibili alle radiazioni infrarosse, che sono invece scarsamente assorbite dall'atmosfera.

L'assorbimento atmosferico è ancora quello che indebolisce enormemente le radiazioni ultraviolette che provengono dal Sole. Il Sole infatti a causa della sua elevatissima temperatura (circa 5000°) è una sorgente estremamente intensa di radiazioni ultraviolette e a grandi altezze, dove l'assorbimento atmosferico è poca cosa, i raggi violetti e ultravioletti costituiscono la maggior parte della radiazione solare.

Al di là dell'ultravioletto ha inizio invece lo spettro dei raggi X che inversamente sono tanto meno assorbiti quanto più piccola è la loro lunghezza d'onda. Essi furono scoperti da Röntgen nel 1895 con un tubo a raggi catodici di Lenard e la loro comparsa nel dominio della fisica segna effettivamente una delle più grandi date nella storia di questa.

La natura ondulatoria dei raggi X fu messa in evidenza la prima volta da U. Laue, che fece usare come reticolo interferenziale per questi raggi il minutissimo reticolo naturale d'un cristallo.

Per la generazione dei raggi X si adoperano oggi in generale i così detti tubi di W. Coolidge. Il metodo per ottenerli con questi tubi è ancora quello di Röntgen: raggi catodici (elettroni) lanciati a velocità enormi (vicine a quella della luce) contro delle placche (anticatodi) che li arrestano bruscamente; ma in luogo di ricavare gli elettroni per effetto della scarica residua nell'interno del tubo stesso, come avveniva nei vecchi tubi, in quelli Coolidge gli elettroni sono forniti da un catodo in tutto analogo a quello delle comuni valvole termoioniche.

Anche per la rivelazione dei raggi X sono molto usate le lastre. Dato il forte potere penetrante di questi raggi, si tratta in genere di lastre speciali con gelatine capaci di assorbire il più possibile i raggi stessi. Nelle ricerche fisiche però, e specialmente per misure accurate d'intensità, si usano sovente, per la rivelazione dei raggi X, anche le camere di ionizzazione, utilizzando l'alto potere ionizzante che i raggi presentano.

Vengono infine i raggi elettromagnetici e precisamente i cosiddetti raggi $\gamma$ secondo il nome che fu dato all'inizio degli studî sulla radioattività per distinguerli dalle radiazioni corpuscolari ($\alpha$ e $\beta$) contemporaneamente emesse dalle sostanze radioattive medesime. Essi si comportano in modo del tutto analogo ai raggi X (di piccola lunghezza d'onda) e si rivelano in genere anche in modo analogo. Però anche qui, quando si passi a raggi di brevissima lunghezza d'onda, occorrono particolari precauzioni e accorgimenti, dato il loro grande potere di penetrazione. Per es. si utilizzano delle camere a forte pressione, si ricorre a incidenze radenti per farli diffrangere su cristalli e ancora più spesso si usano dei metodi indiretti per studiarli (per es., lo spettrografo magnetico per i raggi $\beta$ ad essi associati).

Al di là dei raggi $\gamma$ delle sostanze radioattive, nel dominio dei raggi ultragamma, tutti i dati sono ancora molto incerti. La cosiddetta radiazione penetrante o radiazione cosmica, fino a poco tempo fa ritenuta costituita di raggi elettromagnetici di lunghezza d'onda ancora più breve dei raggi $\gamma$, si è dimostrata di un'estrema complessità e mentre si può ora asserire con certezza che essa è in gran parte composta di radiazioni corpuscolari, non si può affermare con altrettanta sicurezza che in essa sia presente, in modo sensibile, anche una radiazione ultragamma.

Se mai, dove si può essere certi che si creino delle radiazioni ultragamma è nelle disintegrazioni artificiali degli elementi, ottenute bombardando questi ultimi con particelle materiali di grandissima energia.

Bibl.: Per le onde: acustiche, elastiche, elettromagnetiche, si vedano trattati di fisica generali quali: H. Bonazza, Cours de physique, Parigi 1926; H. Ollivier, Cours général de physique, ivi 1928; E. Perrucca, Trattato di fisica, Milano 1934; H. Geiger e R. Scheel, Handbuch der Physik, VI, Berlino 1928; VIII, 1927; XX, 1928; e le bibl. delle voci elasticità; elettricità; suono.

28. Onde di traslazione nei liquidi. - Allorquando un volume d'acqua si trova occasionalmente sollevato sul livello generale della massa che lo circonda, esso tende a rientrarvi con un movimento che in idraulica si dice un'onda di traslazione. La sopraelevazione può essere prodotta da varie cause come, ad es., dal moto d'un natante, dall'immissione di un certo volume d'acqua, ecc. L'onda di traslazione si presenta tutta sopraelevata rispetto al suddetto livello generale e in questo si distingue dall'onda di oscillazione cioè periodica (n. 12), nella quale ogni sopraelevazione è accompagnata da un corrispondente avvallamento. Inoltre l'onda di traslazione, a differenza di quelle di oscillazione che si presentano raggruppate, procede sola; ed è questa la ragione per cui da Scott Russell fu chiamata onda solitaria (n. 12), denominazione questa che però, in seguito, venne specificatamente data a una particolare onda di traslazione che ha la proprietà di propagarsi senza deformarsi.

La velocità di propagazione di un'onda di traslazione di altezza a che si produca in un canale con larghezza superficiale l e la cui sezione liquida è $\omega$, può essere espressa, ammettendo ipotesi semplificative, dalla formula:

dove g è il valore dell'accelerazione della gravità. Se il canale ha sezione rettangolare con profondità d'acqua h, la formula (13) si riduce evidentemente alla:

che, trascurando (a/h)2 in confronto dell'unità, si può mettere sotto la forma:

Infine se a, altezza dell'onda, è molto piccola in confronto di h, profondità dell'acqua, la formula si riduce a:

Scott Russell aveva, dalle sue esperienze, dedotto la formula:

Tale formula fu verificata da Bazin ed essa si riduce ugualmente alla (16) se a è trascurabile di fronte a h.

L'energia totale di un'onda di traslazione, ossia il lavoro che essa produrrebbe se la sua massa ritornasse alla quiete, è costante e può essere espressa da:

dove Q è il volume totale dell'onda, $\xi$ l'altezza del centro di gravità del volume Q al disopra della superficie libera, $\rho$ la densità dell'acqua. Se si ammette che sia $\xi$ =- a/2, la formula (18) diventa:

L'onda di traslazione produce una corrente la cui velocità viene espressa dalla formula:

dove V è la velocità di propagazione dell'onda prima determinata.

29. Onde di marea fluviale. - Esaminiamo ora un caso assai importante di produzione di onde in un fiume o canale sfociante in un mare che presenti apprezzabili altezze di marea. Tali onde che si chiamano onde di marea fluviale si possono assimilare alle onde di traslazione, ma si differenziano alquanto nei riguardi della propagazione: e infatti le rispettive formule, prima accennate, alle quali si combina la velocità propria della corrente fluviale, possono valere con qualche approssimazione per le piccole onde che si propagano in un canale relativamente largo e profondo e dove quindi l'influenza delle resistenze è inapprezzabile, ma non per le onde di maree fluviali per le quali l'altezza sia comparabile alla profondità del fiume e che si propagano in alvei sinuosi, ingombrati da banchi e da isole ed evacuanti portate più o meno ragguardevoli.

L'energia di onda di marea, o onda di traslazione, espressa come si è visto da E = $\rho$gQa, viene spesa ogni volta che l'onda vince una resistenza che si oppone alla sua propagazione. Questa perdita darà luogo o ad una diminuzione del volume dell'onda o ad una riduzione dell'altezza, o ai due effetti simultaneamente.

Siccome i fiumi a marea presentano in generale una forma svasata, le variazioni d'ampiezza sono poco importanti per cui le perdite d'energia si manifestano con una diminuzione del volume dell'onda. Quindi si conclude, intanto, che ogni elemento d'onda che rimonta il fiume perde, ad ogni istante, una parte del proprio volume. Questo volume abbandonato è evacuato verso valle non più seguendo le leggi del moto ondulatorio, perché il moto con cui si produce la corrente di ritorno non ha niente di comune col modo di formazione delle onde di traslazione, ma seguendo le leggi del movimento in funzione della pendenza di superficie, proprie del regime fluviale dei corsi d'acqua. Si deduce dunque che un'onda di traslazione che avanza in un fiume procura una corrente diretta verso valle la cui portata rappresenta, ad ogni istante, il volume d'acqua perduto nell'unità di tempo dalla parte dell'onda che ha passato la foce, aumentata della portata dalle acque superiori. È ovvio che il ragionamento fatto vale, oltre che per la sezione di sbocco del fiume, per qualsiasi altra del fiume stesso. Questa corrente diretta verso valle e prodotta dalla resistenza del fiume si chiama corrente di ritorno o controcorrente. Quanto esposto spiega tutti i caratteri della marea fluviale. Si consideri infatti la regione di sbocco del fiume e si segua il fenomeno della marea a partire dalla stanca di bassa marea. Durante l'onda saliente, il livello del mare si alza progressivamente e provoca nel fiume la successione di elementi d'onda di traslazione la cui altezza cresce fino all'istante del colmo nel quale essa è uguale all'altezza della marea. A partire da questo istante l'altezza degli elementi decresce continuamente fino ad annullarsi alla bassa marea. Quest'insieme di elementi d'onda dà forma ad una immensa onda di traslazione che avanza nel fiume con velocità variabili. Se le resistenze non esistessero e se il fiume avesse una lunghezza indefinita, la velocità di propagazione dell'onda e la velocità della corrente sarebbero, in ciascun punto, e rispettivamente date dalle formule:

dove V'' è la velocità propria delle acque del fiume. Ma così non avviene. Nel momento considerato come di partenza a bassa marea, la parte dell'onda marea precedente che si trova ancora nel fiume, evacua verso mare una certa portata che è in rapporto con le resistenze del corpo d'acqua e l'entità delle acque superiori. La corrente a bassa marea così formata si accresce, a marea crescente, del volume delle acque abbandonate dalla parte della nuova onda che si avanza nel fiume. Nello stesso tempo questa stessa onda produce una corrente diretta verso monte che aumenta continuamente e finisce con l'annullare completamente la controcorrente.

In questo momento si verifica la stanca di bassa marea. Per quanto si è detto questa si deve verificare a marea crescente anche quando non vi è deflusso da monte. Il ritardo della stanca di bassa marea sull'ora di questa dipende dalle resistenze del fiume e dall'importanza del deflusso suddetto. A partire dal punto di stanca di bassa marea la corrente di ritorno è meno potente di quella dell'onda di traslazione per modo che la velocità risultante è una velocità di flusso. Questa fase persiste fino a che il volume delle acque abbandonate dall'onda nell'unità di tempo, aumentata dal deflusso proveniente da monte, risulti nuovamente uguale alla portata propria dell'onda di traslazione. A questo momento si forma la stanca di alta marea. La posizione di questa stanca in rapporto al colmo di marea non è neppure essa ben definita. L'esperimento dimostrerebbe però che la stanca di alta marea si produce sempre dopo il punto di colmo della marea, salvo nelle regioni a monte del fiume dove, sotto l'influenza della portata maggiore, la stanca di alta marea precede il colmo. Se ci si avvicina sufficientemente al limite della marea si trova una regione dove le due stanche coincidono e a monte di questo punto non esiste che una corrente di bassa marea e il gonfiamento della marea è costituito interamente dalle acque superiori.

A partire dalla stanca di alta marea gli elementi d'onda si riducono successivamente e quindi la portata della corrente di ritorno è più grande di quella dell'onda di traslazione e perciò il fiume è sottoposto a un regime di corrente di bassa marea. Questa fase persiste fino alla stanca della bassa marea seguente, a partire dalla quale incomincia un nuovo ciclo. Da questa analisi risulta che se sovrapponiamo al diagramma della velocità dell'onda di traslazione quello della velocità della corrente di ritorno si ottiene la figura 10. I punti d'incrocio delle due curve segnano la posizione delle stanche di corrente e le ordinate comprese fra le due curve misurano le velocità corrispondenti al flusso e riflusso.

Bibl.: Bourdelles, Distribution des vitesses suivant la verticale dans les courants de marée, in Ann. des Ponts et Chaussées, 1898; id., Étude sur le régime de la marée dans les estuaires et dans les fleuves, ivi 1900; M. Levy, Leçons sur la théorie des marées, Parigi 1898.

30. Onde marine. - Nel moto ondoso del mare aperto o dei grandi laghi, le particelle liquide degli strati d'acqua superiori, sotto l'impulso del vento, oscillano in un ristretto spazio attorno alla loro posizione d'equilibrio.

Elementi delle onde sono: la lunghezza $\lambda$, l'altezza a, il periodo T e la velocità di propagazione o celerità V = $\lambda$/T. Nelle onde ordinarie, suscitate dai venti, finché esse si svolgono al largo, in grandi profondità, il rapporto $\lambda$/a varia generalmente tra 15 e 20.

Nel Mediterraneo per le onde massime di tempesta raramente sono superati i valori: a = 6 ÷ 7 metri, $\lambda$ = 100 ÷ 150 metri, con T = 98 ÷ 108. Per gli altri mari sono stati calcolati, in base a circa 4000 osservazioni, i seguenti valori medî:

Dalle esperienze risulta che le onde assumono caratteristiche diverse a seconda che si propagano in fondali di profondità (P) maggiore o minore della loro lunghezza. Le due categorie sono indicate, in genere, come onde di oscillazione e onde di traslazione, ma più propriamente si dovrebbero chiamare: onde in acque profonde e onde in acque basse, perché il carattere oscillatorio è proprio di entrambe le categorie, almeno sinché le onde possono liberamente propagarsi, mentre una vera e propria traslazione della massa liquida non si verifica se non quando l'onda, non potendo più liberamente svolgersi, si rompe, trasformandosi in frangente.

Nelle onde di oscillazione (P > $\lambda$), le particelle di tutta la massa descrivono delle orbite circolari, con diametro che va decrescendo a partire dalla superficie fino ad annullarsi praticamente ad una certa profondità, variabile con l'intensità dell'agitazione. Le orbite delle particelle liquide, situate allo stato di riposo su una stessa verticale, sono percorse dalle particelle stesse sempre nel medesimo tempo, così che le velocità orbitali decrescono dalla superficie verso il fondo, come i diametri delle circonferenze percorse.

Cessata la causa generatrice, le onde (libere) continuano a propagarsi, diminuendo d'altezza e di lunghezza: il frangente si verifica quando, per una qualsiasi causa, l'onda si riduca a una lunghezza pari a circa tre volte a e, nel caso di onde delle ordinarie proporzioni propagantisi su fondo acclive, allorché giungono su profondità pari all'incirca ad a.

Nelle onde di traslazione (P > $\lambda$), le orbite descritte dalle particelle liquide sono delle ellissi che vanno sempre più schiacciandosi procedendo dalla superficie verso il fondo, sinché a contatto di questo si riducono ad oscillazioni rettilinee. Cessata la causa generatrice, l'onda si propaga mantenendo invariata la sua forma finché non varia la profondità, ma quando questa diminuisce per acclività del fondo, l'onda si alza, si assottiglia, piega in avanti e frange in fondali tanto maggiori quanto minore è P in confronto a $\lambda$.

Le onde marine si comportano contro gli ostacoli, quali le opere di difesa e le stesse rive, in maniera assai differente secondo il profilo dell'ostacolo: l'energia delle onde, per metà cinetica e per metà potenziale, viene assorbita dai profili acclivi, mentre dalle pareti verticali le onde vengono semplicemente riflesse. Infatti propagandosi l'onda su fondo a debole declivio, come sulle spiagge o sulle scarpate delle gettate, il movimento orbitale delle particelle acquee viene progressivamente contrastato, le orbite allora si deformano; l'onda conseguentemente modifica il suo profilo, alzandosi, accorciandosi, facendosi più acuta e piegandosi in avanti sinché rompe. Tale fenomeno è caratterizzato dalla proiezione in avanti d'una certa massa acquea, animata da una notevole velocità, la quale, incontrando, a sua volta, un ostacolo vi produce degli urti. Col rompersi, l'onda scompare e l'energia che possedeva va consumata negli urti, negli attriti, nelle azioni distruggitrici delle opere. Allorché invece un'onda incontra una parete verticale eretta su alti fondali, essa vi si riflette semplicemente secondo le leggi usuali dei movimenti vibratorî: l'onda subisce solo un cambiamento di direzione, la sua energia cinetica si trasforma in energia potenziale, e sulla parete si esercitano sforzi essenzialmente statici variabili con le caratteristiche dell'onda incidente, ma, cosa fondamentale senza che l'onda liberi energia. Il fatto più saliente, facilmente osservabile, è la deformazione che l'onda subisce con aumento della sua altezza e sopraelevazione del suo livello medio, in contatto con la parete. L'esperienza prova che le onde si riflettono quasi integralmente su pareti inclinate rispetto alla verticale fino a 45°. Per inclinazioni più deboli si produce il frangente.

Le onde riflesse dalla parete verticale, tornando indietro, interferiscono con le onde seguenti. Ne risultano interferenze, in vario senso, nel movimento orbitale e una risacca (fr. clapotis) che può essere violenta ed estendersi sino a più centinaia di metri dalla parete colpita. Le onde riflesse, propagandosi contro il vento, presto si smorzano.

Lo studio del moto ondoso, agli effetti delle azioni esercitate sulle opere di difesa, è stato condotto, oltreché con ricerche matematiche, anche con ricerche sperimentali. Le due teorie più usate nelle applicazioni idrauliche sono quella trocoidale, matematicamente rigorosa, enunciata dal von Gerstner e dal Rankine, e quella cosiddetta del flutto di fondo di P. Cornaglia. (Per la teoria trocoidale, v. sopra, n. 12).

Secondo la teoria del Cornaglia, il movimento ondoso superficiale genera rasente al fondo, quando questo è in pendenza, un movimento alternato, ora nel senso in cui apparentemente avanzano le onde (diretto), e ora in senso opposto (inverso), da lui chiamato flutto di fondo. Il flutto diretto corrisponde al dorso dell'onda e l'inverso al cavo. L'onda superficiale è considerata dal Cornaglia come una sinusoide e le orbite descritte dalle particelle liquide sono da lui ritenute ellissi, aventi l'asse maggiore verticale e ampiezza decrescente dalla superficie verso il fondo, in contatto col quale il movimento si annullerebbe.

Si è cercato d'estendere la teoria trocoidale allo studio dell'onda riflessa o d'interferenza, nell'ipotesi di profondità indefinita del fluido (Barré de Saint-Venant e Flamant). Il risultato più importante al quale si giunge è il seguente: allorché due onde di uguale periodo, lunghezza ed altezza e di senso contrario interferiscono, la superficie libera del liquido diventa una trocoide d'altezza doppia di quella dell'onda incidente. Con metodi analoghi si è estesa la teoria allo studio del fenomeno della riflessione dell'onda contro una parete verticale in profondità finita (Sainflou). Si arriva anche in questo caso al risultato che l'altezza dell'onda di interferenza, ove l'investimento avvenga in direzione ortogonale, è uguale a circa il doppio di quella dell'onda generatrice.

Numerosissime sono le ricerche sperimentali istituite per stabilire la grandezza delle azioni esercitate dalle onde marine sulle opere di difesa. Fin dal 1886 si è cercato di determinare la grandezza delle pressioni direttamente mediante dinamometri (Stevenson, Gaillard), o indirettamente deducendole dall'altezza dei getti sprigionanti si dalle masse d'acqua nel primo momento dell'urto (Cornaglia). Tali getti d'acqua sono dovuti alle pressioni che si sviluppano sulla superficie investita e dipendono, oltre che dalla velocità, dalla grandezza e dalla forma della vena per rapporto all'estensione e forma dell'ostacolo, e dall'elasticità della parete urtata.

Si è anche cercato di calcolare le pressioni medie esercitate nell'urto, tenendo conto delle lesioni da questo determinate nelle opere investite, lesioni che però sono conseguenza della ripetizione degli urti, i quali intervengono in modo ignoto e con effetti differenti secondo il tipo di struttura sollecitata.

Solamente in questi ultimi anni (Congresso internazionale di navigazione, Cairo 1926), si è cercato d'istituire un programma razionale di ricerche, onde poter procedere con uniformità di metodi e di criterî, in diverse località, alla misura diretta delle sollecitazioni cui il mare sottopone le opere marittime. A tale scopo a Genova, ad opera del Consorzio autonomo del porto, sono state iniziate ricerche sistematiche (Albertazzi, Levi). Speciali apparecchi piezometrici sono stati installati a differenti profondità e altezze sulla parete esterna della diga Principe Umberto. Dalle osservazioni, condotte per oltre tre anni, è risultato che, nel caso di onde investenti la muraglia pressoché normalmente, e conservanti innanzi ad essa carattere oscillatorio (Coen-Cagli): 1. gli sforzi orizzontali eccessi, positivi o negativi, di pressione rispetto allo stato di riposo raggiungono il loro valore massimo in prossimità del livello di riposo delle acque, diminuendo molto rapidamente al disopra di esso e al disotto con legge tanto più lenta, quanto minore è il rapporto della profondità dell'acqua alla lunghezza dell'onda; 2. lo sforzo massimo, incomparabilmente minore di quello che le stesse onde eserciterebbero se investissero la parete in forma di frangente, è di poco inferiore (entro i limiti, almeno, delle osservazioni finora raccolte) alla pressione idrostatica corrispondente all'altezza dell'onda al largo della parete; 3. lo sforzo diminuisce verso l'alto tanto rapidamente da annullarsi praticamente a pochi metri d'altezza sul livello di riposo, anche se l'onda, in contatto con la parete, ne sorpassi più o meno il ciglio superiore; 4. con la profondità di metri 15, nella quale corre, a Genova, il tratto di diga cui sono applicati gli apparecchi piezometrici (nel rapporto di circa 1/10 rispetto alla lunghezza, al largo, delle previste massime onde ordinarie di tempesta) lo sforzo esercitato contro la parete verticale diminuisce in profondità, nel caso di onde delle ordinarie proporzioni, con legge tale da ridursi, al piede della parete, ad un valore medio assai prossimo alla metà del massimo, mentre esso risulta sensibilmente uniforme ed eguale a quest'ultimo, nel caso di onde eccezionalmente lunghe rispetto alla loro altezza. (V. tavv. LV-LVIII).

Bibl.: A. Albertazzi, Recenti esperienze sulle azioni dinamiche delle onde contro le opere marittime, in Annali dei Lavori pubblici, 1932; Airy, Tides and Waves, in Encycl. Metrop., 1842; E. Coen-Cagli, Lezioni di costruzioin marittime, Padova 1928; id., Note sulle opere marittime (redatte per il Colombo, Manuale dell'ingegnere, ed. 1933 e completate dall'autore); id., id. sulle condizioni di stabilità dei moli a parete verticale, in Annali dei Lavori pubblici, 1934, fasc. 6°; P. Cornaglia, Sul regime delle spiagge e sulla regolazione dei porti, Torino 1891; Gaillard, Waves, Action in relation to Engineering Structures, Washington 1835; F. J. von Gerstner, Théorie des vagues (trad. di Saint-Venant), 1887; 1° semestre Annales des Ponts et Chaussées; Levi, Pressioni esercitate dal mare contro le dighe a parete verticale, in L'ingegnere, 1934; Sainrtflou, Essai sur les digues maritimes, in Annales des Ponts et Chaussées, 1928; Saint-Venant e Flamant, De la houle et du clapotis, in Ann. des Ponts et Chaussées, 1898, 1° sem.; T. Stevenson, The construct. of Harbours, Edimburgo 1886; F. Volterra, L'azione delle onde sulle opere di difesa del tipo a parete verticale, in Atti del R. Istituto veneto di scienze, lettere ed arti, XC, parte 2ª.

31. Onde atmosferiche. - Nell'andamento dei fenomeni atmosferici si riscontrano delle fluttuazioni, ossia degli spostamenti rispetto al valore medio, la qual cosa ha fatto sempre pensare all'esistenza d'una periodicità, più o meno approssimata nelle condizioni atmosferiche. Questa presunta esistenza d'un periodo ha fatto adoperare il nome di onde per designare fluttuazioni dei varî elementi meteorologici.

L'elemento che è stato più ampiamente studiato e in cui pare che il carattere ondoso esista, benché molto complicato, è la pressione. Si ritiene infatti che le variazioni della pressione atmosferica, quali sono mostrate da una curva barografica, non siano così irregolari come appaiono al primo aspetto, ma possono invece considerarsi come la risultante di varie oscillazioni regolari simultanee, generalmente di tipo smorzato e aventi ciascuna un periodo proprio.

La ricerca delle varie onde costituenti una curva barometrica e la loro ricostruzione in vera forma, grandezza e posizione costituisce il problema della cimanalisi (F. Vercelli) e si sono ideati varî metodi per la sua risoluzione. Le principali onde bariche di cui si è constatata l'esistenza sono:

Onda semidiurna. - Presenta il periodo di 12 ore ed è costante e regolare nelle regioni intertropicali: si presenta anche nettissima nei barogrammi delle regioni extratropicali quando questi non sono fortemente perturbati. La teoria di quest'onda è stata elaborata da molti studiosi, ma si devono a M.-Margules le conclusioni più interessanti, benché non ancora definitive.

Onda diurna. - Ha il periodo di 24 ore ed è di natura termica, ossia legata all'andamento diurno della temperatura.

Onda di 2-3 giorni. - È molto frequente, specialmente in inverno, e presenta alle nostre latitudini ampiezza piuttosto piccola, mentre appare molto ampia ad alte latitudini.

Onda di circa 4 giorni. - Si riconosce molto ficilmente perché si presenta in generale poco perturbata e spesso molto persistente, sicché si può meglio sfruttare per la previsione.

Onda di 8 giorni. - Si presenta spesso isolata e pura, può avere ampiezza talvolta piccola, tal'altra grandissima; è in generale soggetta a smorzamenti rapidissimi e a brusche cessazioni.

Onda di 16 giorni. -È meno frequente delle precedenti; spesso assume notevole ampiezza e persistenza.

Onda di 19-20 giornii. - È molto frequentte e persistente, specie alle latitudini compresea fra 45° e 65°.

Onda di circa 32 giorni. - Prevale nella stagione invernale e raggiunge talvolta ampiezze molto grandi. I valori dei periodi trovati, che approssimativamente si possono indicare con la successione 2, 4, 8, 16, 32, sarebbero quindi in progressione geometrica, come fu segnalato dal sismologo giapponese Omori sin dal 1908.

Secondo il Vercelli le forti variazioni invernali non sono dovute a onde caratteristiche di tale epoca; i periodi proprî delle oscillazioni barometriche non mutano nel corso dell'anno, solo variano le ampiezze.

Secondo alcuni autori le onde bariche sarebbero onde mobili propagantisi lungo i paralleli da ovest a est; secondo altri si tratterrebbe di onde stazionarie aventi centri di pulsazione di segno contrario, con movimenti di altalena. La causa delle onde bariche si può ricercare nella troposefera (movimenti avvettivi di masse d'aria di diversa origine, impulsi termici dovuti alla presenza dei mari e dei continenti, ecc.) ovvero nella stratosfera (onde gravitazinali al limite tropo-stratosferico, onde stratosferichc di natura non ancora ben definita, ecc.).

Bibl.: P. Mildner, Über Luftdruckwellen, Lipsia 1926; M. Margules, Luftbewegungen in einer rotierenden Sphäroidschale, Vienna 1890; E. Palmén, Über die Natur der Luftdruckschwankungen in höheren Schichten, Lipsia 1928; L. Weickmann, Das Wellenproblem der Atmosphäre, Brunswick 1927; id., Wellen im Luftmeer, Lipsia 1924; F. Vercelli, Oscillazioni periodiche, ecc., Montecassino 1916; id., Cimanalisi e applicazioni, Roma 1928; F. Castriota, Sul problema delle onde bariche, in Annali Ufficio presagi, Roma 1931.

%Alcune citazioni bibliografiche:  \citet{bosellini_scienze_2013,hess_mcknights_2013,walker_halliday_2018,telford_applied_1990}.
%Alcune citazioni bibliografiche:  \citet{bosellini_scienze_2013,baggio_note_1969}.
%\citep{turcotte_geodynamics_2014,turcotte_geodynamics_2002}

\bibliographystyle{sgi}
\begin{small}
\bibliography{appunti} 
\end{small}

\end{document}