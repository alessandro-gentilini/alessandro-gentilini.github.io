\documentclass[a4paper]{article}
\usepackage[T1]{fontenc}
\usepackage[utf8]{inputenc}
\usepackage[italian]{babel}

\usepackage[hidelinks]{hyperref}
\usepackage[italian]{isodate}
\usepackage{natbib}

\usepackage{newpxtext}
\usepackage{euler}

\usepackage{amsmath}

\usepackage{quoting}

\begin{document}
\title{Ghiaccio galleggiante}
\author{Alessandro Gentilini}
\maketitle

\cite{grotzinger_understanding_2014} scrivono a pagina 599:

\begin{quoting}
    Ice shelves, like icebergs, float on ocean waters. When they melt,
    there is no change in sea level for the same reason that, when
    ice cubes in your drink melt, the level of the liquid in your glass
    doesn't change.
\end{quoting}

Allo stesso modo \cite{hess_mcknights_2013}, a pagina 550, scrive:

\begin{quoting}
Although the loss of ice shelves does not raise global sea level 
(for the same reason that a floating ice cube does not raise the level of the water in a glass as it melts), 
changing ice shelves can trigger a change in the flow of land-based ice off the continent.
\end{quoting}

Le spiegazioni in \citet{noerdlinger_melting_2007,jenkins_melting_2007} secondo me non sono valide per la didattica.

Lo è invece \cite{lan_does_2010} anche se non è chiarissima la sequenza dei passaggi per giungere alla formula principale (6), 
pertanto qui di seguito metto il mio procedimento.

Guardate la figura -quale?- e convincetevi del fatto che se il volume occupato dall'intero iceberg sciolto, 
che indichiamo con $V_W$, occupasse esattamente lo stesso volume occupato dalla parte sommersa dell'iceberg, 
che indichiamo con $V_S$, allora il livello del mare non crescerebbe rispetto al livello che si ha con 
l'iceberg galleggiante.

Si vuole quindi trovare la differenza tra il volume dell'acqua ottenuto dal completo scioglimento dell'iceberg, indicato con $V_W$,
e il volume della parte sommersa dell'iceberg solido, parte indicata con $V_S$, se tale differenza sarà nulla allora, per quanto
detto poco fa, non si avrà innalzamento del livello del mare.

, si ricava quindi $V_W$ dalla (4):

\[V_W=\frac{\rho_i}{\rho_W}V_i\]

e si ricava $V_S$ dalla (3):

\[V_S=\frac{\rho_i}{\rho_S}V_i\]

quindi la differenza è

\[V_W-V_S=\frac{\rho_i}{\rho_W}V_i-\frac{\rho_i}{\rho_S}V_i=(\frac{\rho_i}{\rho_W}-\frac{\rho_i}{\rho_S})V_i\]

che corrisponde alla formula (6) che è più comodo riscrivere come:

\[V_W-V_S=(\frac{1}{\rho_W}-\frac{1}{\rho_S})\rho_i V_i\]



Se l'acqua di mare fosse dolce allora si avrebbe $\rho_S=\rho_W$ e quindi

% \begin{lstlisting}
%     \begin{figure}
%         \centering
%         \includegraphics[width=\columnwidth]{IMG_6746} 
%         \caption{Esempio di figura.}
%         \label{fig:stereonet1}
%     \end{figure}
%     \end{lstlisting}

Ghiaccio puro senza bolle d'aria.

No galleggiamento dall'aria.

no freddo cambia densità

% https://www.reddit.com/r/askscience/comments/9oertl/do_melting_icebergs_raise_sea_levels/?show=original

% https://blogs.reading.ac.uk/weather-and-climate-at-reading/2015/melting-of-floating-ice-and-sea-level/






\bibliographystyle{sgi}
\begin{small}
\bibliography{ghiaccio_galleggiante} 
\end{small}

\end{document}